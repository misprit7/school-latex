\documentclass[letterpaper, reqno,11pt]{article}
\usepackage[margin=1.0in]{geometry}
\usepackage{color,latexsym,amsmath,amssymb,graphicx,float,listings,tikz,xcolor}
\usepackage{hyperref}

\hypersetup{
colorlinks=true,
linkcolor=magenta,
filecolor=magenta,
urlcolor=cyan,
}

\lstset{
basicstyle=\ttfamily,
columns=fullflexible,
frame=single,
breaklines=true,
postbreak=\mbox{\textcolor{red}{$\hookrightarrow$}\space},
}

\graphicspath{ {images/} }

\begin{document}

\begin{titlepage}
\newgeometry{margin=3cm}
\centering

\vspace*{\stretch{2}}

\Large Investigation of Fourier Optics in Fraunhofer and Fresnel Limits

\normalsize

\vspace{\stretch{1}}

% \normalsize Power Supplies and Voltage Regulators

\vspace{\stretch{0.5}}

\begin{tabular}{ll}
Name & Xander Naumenko \\[2ex]
Student number & 38198354 \\[2ex]
Partner & Renu Rajamagesh \\[2ex]
Lab name & Fourier optics \\[2ex]
Lab station  & L2C \\[2ex]
Assigned TA            & ? \\[2ex]
Lab \#            & 1 \\[2ex]
Notes &  N/A
\end{tabular}

\vspace{\stretch{3}}


\vspace{\stretch{2}}
\end{titlepage}

\begin{abstract}
    In the far field limit or when using a lens, several techniques from Fourier analysis can be applied to optics to give somewhat surprising results. Several experiments were undertaken in this report to investigate these effects. First, the threshold for where Fourier optics are applicable was tested by observing diffraction patterns of a slit in various configuration. Next, a lens was used to observe the fourier transform of a mesh, text and a razor blade. This was used to optically recognize characters in the object plane, with no active computing required.
\end{abstract}



\textcolor{red}{A.}
 

\end{document}
