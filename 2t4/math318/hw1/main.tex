\documentclass[letterpaper, reqno,11pt]{article}
\usepackage[margin=1.0in]{geometry}
\usepackage{color,latexsym,amsmath,amssymb,graphicx,float,listings}
\usepackage{hyperref}

\hypersetup{
colorlinks=true,
linkcolor=magenta,
filecolor=magenta,
urlcolor=cyan,
}

\graphicspath{ {images/} }

\begin{document}
\pagenumbering{arabic}
\title{Math 318 Homework 1}
\date{18/01/23}
\author{Xander Naumenko}
\maketitle

{\noindent\bf Question 1a.} That sample space is listed in table \ref{tab:q1}. The probabilities can be calculated by calculating the number of permutations of that length. 

\begin{table}[htpb]
    \centering
    \caption{Sample space for question 1}
    \label{tab:q1}
    \begin{tabular}{c|c}
    Sequence&Probability\\
    \hline
    TT&$\frac{1}{4}$\\
    HTT&$\frac{1}{8}$\\
    HHTT&$\frac{1}{16}$\\
    THTT\\
    HHHHH&$\frac{1}{32}$\\
    HHHHT\\
    HHHTH\\
    HHHTT\\
    HTHTT\\
    THHTT\\
    HHTHH\\
    HHTHT\\
    HTHHH\\
    HTHHT\\
    HTHTH\\
    THHHH\\
    THHHT\\
    THHTH\\
    THTHH\\
    THTHT\\
    \end{tabular}
\end{table}

{\noindent\bf Question 1b.} Since they are independent events we can sum the probabilities from the previous questions, so we get: 
\[
i=2: \frac{1}{4}
.\]
\[
i=3: \frac{1}{8}
.\]
\[
i=4: \frac{1}{8}
.\]
\[
i=5: \frac{1}{2}
.\]
As expected the probabilities sum to 1. 

{\noindent\bf Question 2.} There are 40 marked frogs, so the odds of getting exactly 14 frogs are 

\[
    L(n)=\frac{{40\choose 14}{n-40\choose 36}}{{n\choose 50}}=\frac{40!(n-40)!50!(n-50)!}{14!26!36!(n-76)!n!}
.\]

As the hint suggests, we can solve $\frac{l(n)}{L(n-1)}= 1$ to find where $L(n)$ is at its maximum. 
\[
\frac{L(n)}{L(n-1)}=\frac{(n-40)(n-50)}{n(n-76)}=1
\]
\[
\implies (n-40)(n-50)=n^2-76n\implies n=\frac{1000}{7}
.\]
Thus $L(n)$ is increasing for $n<\frac{1000}{7}$ and decreasing for $n>\frac{1000}{7}$. Thus either $n=142$ or $n=143$ maximizes $L(n)$, computing them shows that $n_*=143$ is the most likely. 

{\noindent\bf Question 3a.} Plugging in values: 
\[
P_1=\frac{3}{51}\cdot \frac{48}{50}\cdot \frac{44}{49}\cdot \frac{40}{48}=0.42
.\]
\[
    P_2=\frac{{13\choose 2}{4\choose 2}{4\choose 2}\cdot 11\cdot 4}{{52\choose 5}}=0.047
.\]
{\noindent\bf Question 3b.} Same as before: 
\[
    P_1=\frac{6\cdot 5\cdot 4\cdot 3\cdot {5\choose 2}}{6^5}=0.46
.\]
\[
    P_2=\frac{6\cdot 5\cdot 4\cdot {5\choose 2}\cdot {3\choose 2}}{6^5}=0.23
.\]

{\noindent\bf Question 4a.} Using number of possibilities: 
\[
    p_n=\frac{{2n\choose n}}{2^{2n}}=\frac{(2n)!}{n!^22^{2n}}
.\]

{\noindent\bf Question 4b.} 
\[
    p_{n+1}/p_n=\frac{(2n+2)(2n+1)}{(n+1)^22^2}= \frac{(n+1)(n+\frac{1}{2})}{(n+1)^2}\leq 1
.\]

The ratio of successive elements being less than one means that the next term is smaller than the previous, so the sequence is decreasing. 

{\noindent\bf Question 4c.} Using sterling's approximation: 
\[
p_n=\frac{\sqrt{4\pi n} \left( 2n /e \right)^{2n} }{2\pi n \left( n /e \right)^{2n}2^{2n}}=\frac{1}{\sqrt{\pi n}}\implies\alpha = \frac{1}{\sqrt{\pi} }
.\]

{\noindent\bf Question 5a.} There are $n$ balls and $m-1$ lines to place so the number of configurations is just the number of ways to arrange this which is ${n+m-1\choose m-1}$. 

{\noindent\bf Question 5b.} Here we are just choosing which of the m urns to put our n balls in which is the definition of the choice function, which is just ${m\choose n}$. 

{\noindent\bf Question 6.} See the code below. 

\begin{lstlisting}[language=Python]
import random
import math
import matplotlib.pyplot as plt

for days in {365, 669}:
    def birthday(n):
        return len({random.randint(1, days) for i in range(n)}) != n

    N = list(range(2, 61))
    X = []
    for n in N:
        matches = 0
        runs = 0
        for i in range(1000):
            matches += birthday(n)
            runs += 1
        X.append(matches / runs)

    Y = [1 - math.factorial(days) / math.factorial(days - n) / days**n for n in N]

    plt.plot(N, X)
    plt.plot(N, Y)
    plt.show()
\end{lstlisting}

\end{document}
