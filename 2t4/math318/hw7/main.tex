\documentclass[letterpaper, reqno,11pt]{article}
\usepackage[margin=1.0in]{geometry}
\usepackage{color,latexsym,amsmath,amssymb,graphicx,float,listings,tikz}
\usepackage{hyperref}

\hypersetup{
colorlinks=true,
linkcolor=magenta,
filecolor=magenta,
urlcolor=cyan,
}

\graphicspath{ {images/} }

\begin{document}
\pagenumbering{arabic}
\title{Math 318 Homework 7}
\date{16/03/23}
\author{Xander Naumenko}
\maketitle

{\medskip\noindent\bf Question 1a.} The definition of a characteristic function is $\phi_{X}(t)=E[e^{itX}]$. Expanding this, we get: 
\[
\phi_{X}(t)=E[e^{itX}]=\int_{-\infty}^{\infty}f_X(x) e^{itx}dx=\sum_{x=-\infty}^{\infty}p_X(x) e^{itx}=\sum_{x=-\infty}^{\infty}p_X(x) \left( \cos(tx)+i\sin(tx) \right) 
.\]
The only $t$ dependence is in the $\sin(tx)$ and $\cos(tx)$ terms which are both $2\pi$ periodic (since $x$ is an integer), so the characteristic function of discrete variables is $2\pi$-periodic. 

% {\medskip\noindent\bf Question 1b.} Let $Y=$

% \[
%     E[Y]=\frac{1}{i}\phi_Y'(0)=\int_{-\infty}^{\infty}yf_Y(y)dy=\frac{1}{i}\phi_Y'(2\pi)=\int_{-\infty}^{\infty}yf_Y(y)e^{2\pi iy}dy
% .\]
% \[
% \phi_X(0)=\int_{-\infty}^{\infty}f_X(x)dx=\phi_X(2\pi)=\int_{-\infty}^{\infty}f_X(x)e^{2\pi ix}dx
% .\]

{\medskip\noindent\bf Question 1b.} Consider $\phi(2\pi)$. By definition and the periodicity of $X$, this is
\[
    \phi(2\pi)=E[e^{2\pi iX}]=\phi(0)
.\]

{\medskip\noindent\bf Question 2.} Expressing as an integral: 
\[
    E[X^3]=\int_{-\infty}^{\infty}\int_{-\infty}^{\infty}(u+v)^3f_U(u)f_V(v)dudv=\iint (u^3+3u^2v+3uv^2+v^3)f_U(u)f_V(v)dudv
\]
\[
    =E[U^3]+E[V^3]+3E[U^2]E[V]+3E[U]E[V^2]=E[U^3]+E[V^3]
.\]
Similarly for $E[X^{4}]$: 
\[
    E[X^4]=\int_{-\infty}^{\infty}\int_{-\infty}^{\infty}(u+v)^4f_U(u)f_V(v)dudv=\iint (u^4+4u^3v+6u^2v^2+4uv^3+v^3)f_U(u)f_V(v)dudv
.\]
\[
    =E[U^4]+E[V^4]+6E[U^2]E[V^2]+4E[U^3]E[V]+4E[U]E[V^3]=E[U^4]+E[V^4]+6E[U^2]E[V^2]
.\]

{\medskip\noindent\bf Question 3a.} Let the entries of $A$ be denoted by $A_{ik}$, where $i$ is the row and $k$ is the column. Then by the definition of matrix multiplication,
\[
    Y_i=\sum_{k=1}^{n}A_{ik}X_k
.\]
The distribution of a sum of normal variables is also a normal variable with the sum of mean and variance added, so $Y_i=N(0,n)$. 

{\medskip\noindent\bf Question 3b.} Computing covariance
\[
    \text{Cov}(Y_i,Y_j)=E[Y_iY_j]-E[Y_i]E[Y_j]=E\left[\left(\sum_{k=1}^{n}A_{ik}X_k\right)\left( \sum_{k=1}^{n}A_{jk}X_k \right) \right]
\]
\[
    =E\left[\sum_{k=1}^{n}\sum_{l=1}^{n}A_{ik}A_{jl}X_kX_l\right]=\sum_{k=1}^{n}\sum_{l=1}^{n}A_{ik}A_{jl}E[X_kX_l]=\sum_{k=1}^{n}A_{ik}A_{jk}E[X_k^2]
\]
\[
=A_i\cdot A_j
\]
where $A_i$ and $A_j$ are the $i$th and $j$th row vector of $A$ respectively.

{\medskip\noindent\bf Question 3c.} 

{\medskip\noindent\bf Question 4.} For the following sections this code was used to generate the figures:

\begin{lstlisting}
import numpy as np
import random
import matplotlib.pyplot as plt
from tqdm import tqdm

n = 1000000
sims = 1000
ps = (0.5, 0.51, 0.502)

figa, axa = plt.subplots(3)
for j, p in enumerate(ps):
    X = [0]
    for i in range(n-1):
        X.append(X[i] + (-1 if random.random() > p else 1))
        
    axa[j].plot(list(range(n)), X)
    axa[j].set_title(f'p={p}')

plt.show()

figb, axb = plt.subplots(3)
figc, axc = plt.subplots(3)
for j, p in enumerate(ps):
    T = np.array([n]*sims)
    for sim in tqdm(range(sims)):
        X = 0
        for t in range(n):
            X += -1 if random.random() > p else 1
            if X == 0:
                T[sim] = t
                break

    axb[j].hist(T, bins=np.arange(0, n + 10000, 10000))
    axb[j].set_title(f'p={p}')

    T.sort()
    F = []
    s = 0
    for t in range(n):
        while s < len(T) and T[s] <= t:
            s += 1
        F.append((sims-s)/sims)

    axc[j].loglog(np.arange(n), F)
    axc[j].set_title(f'p={p}')

plt.show()
\end{lstlisting}

{\medskip\noindent\bf Question 4a.} See figure \ref{fig:4a}. 

\begin{figure}[htpb]
    \centering
    \includegraphics[width=0.8\textwidth]{4a}
    \caption{Graph for question 4a.}
    \label{fig:4a}
\end{figure}

{\medskip\noindent\bf Question 4b.} See figure \ref{fig:4b}. 

\begin{figure}[htpb]
    \centering
    \includegraphics[width=0.8\textwidth]{4b}
    \caption{Histogram for question 4b.}
    \label{fig:4b}
\end{figure}

{\medskip\noindent\bf Question 4c.} See figure \ref{fig:4c}.

\begin{figure}[htpb]
    \centering
    \includegraphics[width=0.8\textwidth]{4c}
    \caption{Graphs for question 4c.}
    \label{fig:4c}
\end{figure}

{\medskip\noindent\bf Question 4d.} From the plots it seems that $P(T>n)=0$ as $n\to \infty$. 

{\medskip\noindent\bf Question 5.} The following code was used for this question: 
\begin{lstlisting}
import numpy as np
import matplotlib.pyplot as plt
import random
from tqdm import tqdm

n = 1000000
sims = 1000
directions = [np.array(d) for d in [(1, 0), (0, 1), (-1, 0), (0, -1)]]

S = np.empty((n, 2))

for i in range(n-1):
    S[i+1] = S[i] + random.sample(directions, 1)

plt.scatter(S.T[0], S.T[1], c=np.arange(0, n), cmap="rainbow", s=0.2)

plt.show()


n_steps = []
for sim in tqdm(range(sims)):
    S = np.empty((n, 2))

    n_steps.append(n)
    for i in range(n-1):
        S[i+1] = S[i] + random.sample(directions, 1)
        if S[i+1][0] == 0 and S[i+1][1] == 0:
            n_steps[-1] = i+1
            break

plt.hist(n_steps)

print(n_steps)
print(len([x for x in n_steps if x == n]))

plt.show()
\end{lstlisting}

{\medskip\noindent\bf Question 5a.} See figure \ref{fig:5a}

\begin{figure}[htpb]
    \centering
    \includegraphics[width=0.8\textwidth]{5a}
    \caption{Scatter plot for 5a.}
    \label{fig:5a}
\end{figure}

{\medskip\noindent\bf Question 5b.} The number of walks that failed to return to 0 is . This is consistent with the fact that random walks are recurrent, as we are terminating the walks at a finite threshold, in an infinite walk the probability of reaching 0 again is 1. 

{\medskip\noindent\bf Question 5c.} See figure \ref{fig:5b}.

\end{document}
