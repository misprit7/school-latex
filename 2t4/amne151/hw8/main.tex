\documentclass[letterpaper, reqno,11pt]{article}
\usepackage[margin=1.0in]{geometry}
\usepackage{color,latexsym,amsmath,amssymb,graphicx,float,listings,tikz}
\usepackage{hyperref}

\hypersetup{
colorlinks=true,
linkcolor=magenta,
filecolor=magenta,
urlcolor=cyan,
}

\graphicspath{ {images/} }
\usepackage{setspace}
\doublespacing

\begin{document}
\pagenumbering{arabic}
\title{ANNE Close Reading 8}
\date{10/02/23}
\author{Xander Naumenko}
\maketitle

{\bf Question:} How does this text understand slavery or servitude? What does it tell us about relationships marked by such an imbalance of power?

\medskip

The Roman play {\em Amphitruo}, written by the playwright Plautus, portrays a society where slavery is commonplace, and slaves are treated as property rather than human beings. The play presents a complex understanding of slavery, showcasing both its oppressive nature of slavery and the complex relationships that can arise between masters and slaves. Throughout the play, the audience is presented with characters who are either slaves or masters, and the power dynamic between them is a central theme.

\medskip

One of the most striking aspects of Amphitruo is the way in which it portrays the power imbalance between masters and slaves. The slaves in the play are consistently portrayed as being subservient to their masters, and their actions are often dictated by their masters' wishes. This dynamic is portrayed in the character of Sosia, Amphitruo's slave, who is constantly at the mercy of his master's whims. At the same time, however, the play also showcases the slaves' cunning and resourcefulness in dealing with their masters. Sosia's dialogues often center around how to deal with the possibility of his Amphitruo punishing him, this suggests that while slavery is an inherently oppressive institution, the slaves are not without agency and can find ways to assert themselves even in the face of their masters' power.

\medskip

Another key theme in Amphitruo is the impact of slavery on relationships between masters and slaves. The play explores the complex and often fraught relationships that can arise between individuals who are marked by such a profound imbalance of power. At times, Sosia and Amphitruo discuss with one another and help each other get a better understanding of the confusing doubling that's happening at the gods' whims. However even in these mutually beneficial interactions there is the constant looming threat of physical violence. For example, Amphitruo, in his anger at Sosia's apparently inconsistent story, mentions ``I’ll cut your tongue for this, you wretch.'' (line 555) This, as well as many other similar passing threats of violence show that even when the master/slave relationship isn't entirely antagonistic, there is always the subtext of power that fundamentally warps the interaction. 

\medskip

Overall, Amphitruo presents a nuanced understanding of slavery, highlighting both the oppressive nature of the institution and the complex relationships that can arise between masters and slaves. The play suggests that while slavery is an inherently unjust and oppressive institution, slaves are not without agency and can find ways to assert themselves even in the face of their masters' power. At the same time, however, the play also shows how the power dynamic between masters and slaves can lead to exploitation. 

\end{document}
