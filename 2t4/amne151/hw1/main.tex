\documentclass[letterpaper, reqno,11pt]{article}
\usepackage[margin=1.0in]{geometry}
\usepackage{color,latexsym,amsmath,amssymb,graphicx,float}
\usepackage{hyperref}
\usepackage[
backend=biber,
style=alphabetic,
sorting=ynt
]{biblatex}
\addbibresource{sources.bib}

\hypersetup{
colorlinks=true,
linkcolor=magenta,
filecolor=magenta,
urlcolor=cyan,
}

\graphicspath{ {images/} }

\begin{document}
\pagenumbering{arabic}
\title{AMNE Close Reading Assignment 1}
\date{19/01/23}
\author{Xander Naumenko}
\maketitle


Question to answer: How does this text understand sexuality? What does it tell us about what or who is considered the norm?  

Theogony is an interesting work that reveals much into the views on sexuality of the ancient Greek. Perhaps the best example of this is with Zeus, the king of the gods. Around line 900 there's a fairly long section where the text describes the various goddesses that Zeus lays with. Interestingly, the text calls his various unions marriages despite them clearly not being monogamous affairs. For example ``Zeus, king of the gods, made Metis his wife first'' \cite{theogony}, despite him in the very next paragraph marrying again to Themis. Obviously as a god it's very possible that the norms presented in the work apply to Greek society as a whole, but regardless it's interesting the polyamorous nature of the gods' relationships. 


\medskip

\printbibliography

\end{document}
