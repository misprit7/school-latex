\documentclass[letterpaper, reqno,11pt]{article}
\usepackage[margin=1.0in]{geometry}
\usepackage{color,latexsym,amsmath,amssymb,graphicx, float}
\usepackage{hyperref}

\usepackage[
backend=biber,
sorting=ynt
]{biblatex-chicago}
\addbibresource{sources.bib}

\hypersetup{
colorlinks=true,
linkcolor=magenta,
filecolor=magenta,
urlcolor=cyan,
}
\usepackage{setspace}
\doublespacing

\graphicspath{ {images/} }

\begin{document}
\pagenumbering{arabic}
\title{AMNE 151 Capstone 3\\ \large Option 2: Heroic Research}
\date{March 31st, 2023}
\author{Xander Naumenko, 38198354}
\maketitle

{\noindent\bf Hero Choice: Atalanta}

In the abstract, the story of Atalanta is very much that of a stock Greek hero; she is of noble birth, bravely fights the Calydonian Boar and becomes embroiled in god related drama. Despite this, Atalanta is very unlike most other Greek heros in that most of the key conflict within her stories are defined not by external battles with terrible monsters or gods, but by her inspiring struggle with society itself over her place as a woman warrior.

\medskip

This struggle manifests itself within days of being born. ``At birth her father exposed her; he said he wanted sons, not daughters.''\autocite{Aelian} She is then raised by bears and becomes a devotee of Artemis. In her very first interactions with society she is shunned in a physical sense over her gender. This overt occlusion from the male-glorifying world rears its head once again when she tries to join the Argonauts. It is described how ``she eagerly desired to follow on that quest; but [Jason] of his own accord prevented the maid, for he feared bitter strife on account of her love.''\autocite{Apollonius} Given that she was a disciple of Artemis and her later stories emphasize her virginity it seems likely that such claims of love on Jason's part were based solely on her gender and no small degree of male cockiness. While she eventually does earn her place on the Argonauts, these early foundational struggles have their core conflict intrinsically embedded in her struggle with society for acceptance.

\medskip

The immense hardship brought upon society's disdain for an unchangeable part of personal identity isn't something that is characteristic of most Greek heros. Contrast perhaps the most famous myths of Theseus and Atalanta: the Minotaur and the Calydonian Boar respectively. Upon Theseus's success in killing the Minotaur, ``King Minos was so glad that \ldots he sent him home free into his country, giving to him all the other prisoners of Athens.''\autocite{Plutarch}\footnote{There's no line number's in this text, but it's right at the end in the textbook} Theseus is unambiguously and immediately praised for his feat, there is no question of his background or whether he deserves it. Compare this to the reaction to when Atalanta helps defeat the Calydonian Boar and receives its prized skin: ``the sons of Thestios, thinking scorn that a woman should get the prize in the face of men, took the skin from her, alleging that it belonged to them by right.''\autocite{Bibliotheca} Almost half of this text about her is devoted not to her battle with the boar, but to her difficulties asserting she was responsible.

\medskip

For better or for worse, much of Atalanta's mythology is related to her struggles with society to recognize her achievements as that of a woman. It's worth considering though what this tells us about the view that ancient Greek writers had on gender. On one hand, the fact she must contend with men denying her achievements takes space away from the writers emphasizing the heroism of her deeds. It might be meant to leave a Greek listener with the impression that she is notable only for the fact she managed to accomplish challenges as a woman rather than being a great hero who was female. On the other hand though, such a clear story of a woman facing adversity from Greek society and overcoming it is worth telling in its own right. It shows how Greek writers may have understood the path to becoming an independent woman their culture was a long and arduous one, for a mythical hero or not, but at least its telling reflects that they thought it worthy of consideration.

\pagebreak

\printbibliography

\end{document}
