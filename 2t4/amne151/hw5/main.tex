\documentclass[letterpaper, reqno,11pt]{article}
\usepackage[margin=1.0in]{geometry}
\usepackage{color,latexsym,amsmath,amssymb,graphicx,float,listings,tikz}
\usepackage{hyperref}

\hypersetup{
colorlinks=true,
linkcolor=magenta,
filecolor=magenta,
urlcolor=cyan,
}

\graphicspath{ {images/} }
\usepackage{setspace}
\doublespacing

\begin{document}
\pagenumbering{arabic}
\title{ANNE Close Reading 4}
\date{10/02/23}
\author{Xander Naumenko}
\maketitle

{\bf Question:} How does this text understand death? What does it tell us about what happens when someone dies?

Bion's poem {\em Lament for Adonis} offers a glimpse into the how the Greek viewed death, especially with relation to the gods. What's striking is that although the poem itself is nominally a homage to the sadness of Adonis's passing, it focuses almost entirely on the effect of his death on Aphrodite and those around him. The poem states ``Alas for Aphrodite, beautiful Adonis is dead!'' (there are no line numbers in the poem, but in {\em Lament for Adonis} paragraph 12). Adonis is not the main subject. Instead Adonis's death is sad since it deprives Aphrodite of a lover, albeit one very fair. We see this again when paragraph 18 where the poem describes how Aphrodite will never again be able to sleep with the beautiful Adonis, without any mention of the sadness of Adonis himself having died. 

\medskip

While it's clear that the poem approaches grief in a way that focuses more on Aphrodite, why this is has a few interpretations. One is that the sadness is primarily on Aphrodite's part simply because she is a Goddess. Many times we've seen in our previous readings that the matters of mortals are completely beneath those of the gods, and this could just be a representation of this; as the most important individual of the story Aphrodite deserves the most pity even above the one who died. 

\medskip

Another interpretation of the lack of emphasis on Adonis is a differing view that the Greeks had on death. In paragraph 14 Aphrodite refers to how Adonis is on his way to the underworld with Hades, so perhaps with this assumption the prevailing view was that he had simply moved on to the next stage of existence. Although it certainly isn't ideal that he can't interact with the living, it's not something intrinsically sad for him, and it is those still living that are deprived of their interactions with Adonis that have the most claim to sadness. 

\end{document}
