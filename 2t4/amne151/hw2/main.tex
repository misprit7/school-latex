\documentclass[letterpaper, reqno,11pt]{article}
\usepackage[margin=1.0in]{geometry}
\usepackage{color,latexsym,amsmath,amssymb,graphicx, float}
\usepackage{hyperref}

\hypersetup{
colorlinks=true,
linkcolor=magenta,
filecolor=magenta,
urlcolor=cyan,
}
\usepackage{setspace}
\doublespacing

\graphicspath{ {images/} }

\begin{document}
\pagenumbering{arabic}
\title{AMNE 151 Close Reading 2}
\date{27/01/23}
\author{Xander Naumenko}
\maketitle

{\bf Question:} How does this text understand "gender" or "sex" differences? What does it tell us about these differences? 

\medskip

One theme throughout many of the primary texts we've analyzed as part of this course is the repeated idea that women often attempt to get what they want by having children to eventually champion their goals, whereas the male gods are often portrayed as doing the fighting themselves. This seems to imply a fairly fundamental difference in gender in the greek view, where women, even godesses, aren't fit for direct fighting and must resort to more convoluted means to get their way. 

\medskip 

Probably the most obvious example of this is with Rhea. Diodorus Siculus Histories, near line 5.70.2, talks about how ``Rhea, upset, \ldots, after she had given birth to Zeus, concealed him in Ide.'' She later helps him defeat his father (and her husband) Cronus by poising Cronus's drink. What sticks out about this is that she is fully capable and willing to directly go into opposition with Cronus; she directly poisons him later. However it's somehow assumed that the only way for her to fully defeat him is with the help of her son, who later goes on to become the most important god in the pantheon. 

\medskip 

Another example of women having children to accomplish their goals is when Gaia creates Typhon in an attempt to dethrone him. In Pseudo-Apollodorus, Bibliotheca line 16.3, this is directly stated: ``The defeat of the Gigantes (Giants) by the gods angered Gaia even more, so she had intercourse with Tartaros and gave birth to Typhon in Cilicia.'' Here it's even more obvious than in Rhea's case that Gaia had her child with a specific goal. When Rhea hid Zeus she also wanted to hide him for intrinsic love for her children in addition to her hope of overthrowing Cronus, but here there is no ambiguity as the author directly states Gaia's intentions. 

\medskip

It's not clear why exactly Rhea and Gaia feel they must do things this way. The simplest view would be to just say that the ancient Greek refused to recognize the strength and independence of women, and certainly to a large degree this is true. There's also a view however that it also shows that the ancient Greek had a view that in some sense women can be held both accountable and responsible for their children's deeds and accomplishments. From that perspective Zeus's overcoming of his father actually reflects Rhea's triumph and is an example of Rhea taking on a independent role in causing societal (in the society of deities) change. Even thinking about things this way, it's rather demeaning to women to assume that most of their successes are actually those of their male children, but at least it gives women a degree of autonomy and agency that is easy to gloss over. 

\medskip

\end{document}
