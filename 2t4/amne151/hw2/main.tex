\documentclass[letterpaper, reqno,11pt]{article}
\usepackage[margin=1.0in]{geometry}
\usepackage{color,latexsym,amsmath,amssymb,graphicx, float}
\usepackage{hyperref}

\hypersetup{
colorlinks=true,
linkcolor=magenta,
filecolor=magenta,
urlcolor=cyan,
}

\graphicspath{ {images/} }

\begin{document}
\pagenumbering{arabic}
\title{AMNE 151 Close Reading 2}
\date{27/01/23}
\author{Xander Naumenko}
\maketitle

{\bf Question:} How does this text understand "gender" or "sex" differences? What does it tell us about these differences? 

One theme throughout many of the primary texts we've analyzed as part of this course is the repeated idea that women often attempt to get what they want by having children to eventually champion their goals, whereas the male gods are often portrayed as doing the fighting themselves. This seems to imply a fairly fundamental difference in gender in the greek view, where women, even godesses, aren't fit for direct fighting and must resort to more convoluted means to get their way. 

Probably the most obvious example of this is with Rhea. Diodorus Siculus Histories, near line 5.70.2, talks about how ``Rhea, upset, \ldots, after she had given birth to Zeus, concealed him in Ide.'' She later helps him defeat his father (and her husband) Cronus by poising Cronus's drink. What sticks out about this is that she is fully capable and willing to directly go into opposition with Cronus; she directly poisons him later. However it's somehow assumed that the only way for her to fully defeat him is with the help of her son, who later goes 

\end{document}
