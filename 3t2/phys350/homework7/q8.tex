\documentclass[letterpaper, reqno,11pt]{article}
\usepackage[margin=1.0in]{geometry}
\usepackage{color,latexsym,amsmath,amssymb,graphicx, float}
\usepackage{hyperref}

\hypersetup{
colorlinks=true,
linkcolor=magenta,
filecolor=magenta,
urlcolor=cyan,
}

\graphicspath{ {images/} }

\newcommand{\RR}{\mathbb{R}}
\newcommand{\CC}{\mathbb{C}}
\newcommand{\ZZ}{\mathbb{Z}}
\newcommand{\QQ}{\mathbb{Q}}
\newcommand{\NN}{\mathbb{N}}
\newcommand{\st}{\text{ s.t.}\ }
\newcommand{\tn}[1]{\textnormal{#1}}
\newcommand{\m}{\textnormal{ m}}
\newcommand{\s}{\textnormal{ s}}
\newcommand{\K}{\textnormal{ K}}
\newcommand{\h}{\textnormal{ h}}
\newcommand{\W}{\textnormal{ W}}
\newcommand{\J}{\textnormal{ J}}
\newcommand{\Pa}{\textnormal{ Pa}}
\newcommand{\mol}{\textnormal{ mol}}
\newcommand{\Hz}{\textnormal{ Hz}}
\newcommand{\kg}{\textnormal{ kg}}
\newcommand{\cm}{\textnormal{ cm}}
\newcommand{\mm}{\textnormal{ mm}}
\newcommand{\N}{\textnormal{ N}}

\begin{document}
\pagenumbering{arabic}
\title{Math 220 Question 8}
\date{November 01, 2021}
\maketitle

{\noindent\bf Question 8.} First we will show the forward definition. Assume $R$ is reflexive and circular, we must show it is symmetric and transitive for it to be a equivalence relation. For symmetric, let $b=a$. Then $(aRb\wedge bRc)\Leftrightarrow (aRa\wedge aRc)\Leftrightarrow aRc\implies cRa$, which gives us symmetry.  Combining the fact that $R$ is symmetric and circular, we get that $aRb\wedge bRc\implies cRa\implies aRc$, which is the definition of transistivity. 

For the backwards direction, assume that $R$ is an equivalence relation and we will show that $R$ is reflexive and circular. Reflexive is given automatically by the definition of an equivalence relation. To show circular, similarly to before we combine the transitive and symmetric property of $R$: 

$$
    aRb\wedge bRc\implies aRc\implies cRa\implies R\text{ is circular}
$$

Thus any relation $R$ is an equivalence relation if and only if it is reflexive and circular. $\square$
 

\end{document}