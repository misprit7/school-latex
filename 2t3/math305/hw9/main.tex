\documentclass[letterpaper, reqno,11pt]{article}
\usepackage[margin=1.0in]{geometry}
\usepackage{color,latexsym,amsmath,amssymb,graphicx, float}
\usepackage{hyperref}

\hypersetup{
colorlinks=true,
linkcolor=magenta,
filecolor=magenta,
urlcolor=cyan,
}

\graphicspath{ {images/} }

\begin{document}
\pagenumbering{arabic}
\title{MATH 305 Homework 9}
\date{04/04/22}
\author{Xander Naumenko}
\maketitle

\noindent
1. (20) Compute the Laurent series for

(a) $ \frac{1}{ z(z+2)}, 1<|z-1|<3$

Partial fractions: 
\[
\frac{1}{z(z+2)}=\frac{1}{2z}-\frac{1}{2(z+2)}=\frac{1}{2}\left( \frac{1}{(z-1)}\frac{1}{1+1 /(z-1)}-\frac{1}{3}\frac{1}{1+(z-1) /3} \right) 
.\]
\[
=\frac{1}{2(z-1)}\sum_{n=0}^\infty \frac{(-1)^{n}}{(z-1)^{n}}-\frac{1}{3}\sum_{n=0}^{\infty}\left( (-1)\frac{z-1}{3} \right) ^{n}=-\frac{1}{6}+\sum_{n=1}^{\infty}\frac{(-1)^{n+1}}{2(z-1)^{n}}-\frac{(z-1)^{n}}{2(-3)^{n+1}}
.\]

(b) $\frac{1}{z^2+4}, |z-2i|>4$

\[
\frac{1}{z^2+4}=\frac{1}{4i}\left(\frac{1}{z-2i}-\frac{1}{z+2i}\right)=\frac{1}{4i}\left( \frac{1}{z-2i}-\frac{1}{(z-2i)\left( 1+4i /(z-2i) \right) } \right) 
.\]
\[
=\frac{1}{4i(z-2i)}+\frac{1}{4i}\sum_{n=0}^{\infty}\frac{(-4i)^{n}}{(z-2i)^{n+1}}=(\frac{1}{4i}-1) \frac{1}{z-2i}+\sum_{n=1}^{\infty}\frac{(-1)^{n}(4i)^{n-1}}{(z-2i)^{n+1}}
.\]


\medskip



\noindent
2. (20) Determine the types of all the isolated singularities of the following  functions and compute the residue at each isolated singularity

(a) $\frac{z}{\tan z}$


\[
\frac{z}{\tan z}=\frac{z\cos z}{\sin z}
.\]
This function has singularities for $z=n\pi$. For $z=0$, this is a removable singularity since $\lim_{z\to 0}\frac{z\cos z}{\sin z}=1$, so $Res[\frac{z}{\tan z}; 0]=0$. For $z=n\pi, n\neq 0$ the function has simple poles, which gives:
\[
Res[\frac{z}{\tan z}; n\pi]=\frac{n\pi}{\sec^2 nz}=n\pi
.\]

(b) $ \frac{\cos z}{z^3}$

The only singularity is $z=0$, which is a pole of order 3 (since $\cos(z)\neq 0$, which gives that
\[
Res[\frac{\cos z}{z^3};0]=\frac{1}{2}\frac{d^2}{dz^2}\cos z=-\frac{1}{2}
.\]

(c) $ \frac{\mbox{Log} (z) }{ (z^2+1)^2 }$

The two isolated singularities are at $z=\pm i$, which are simple poles of order 2. Since $\mbox{Log}(\pm i)\neq 0$, we get that the residue is
 \[
Res[\frac{\mbox{Log}(z)}{(z^2+1)^2};\pm i]=\frac{d}{dz}\left( \frac{\mbox{Log}(z)}{(z\pm i)^2} \right)=\frac{\frac{1}{z}(z\pm i)^2-2(z\pm i)\mbox{Log}(z)}{(z\pm i)^4}\bigg|_{z=\pm i}
.\]
\[
=\frac{\mp i (-2i)^2\pm 4i(\mp \frac{\pi}{2}i)}{16}=\pm \frac{1}{4}i+\frac{\pi}{8}
.\]

(d)  $ \frac{e^z}{ 1- \sqrt{z}} $

Since we're using the principle branch the only pole is that $z=1$. Consider the function as follows: 
\[
\frac{e^{z}}{1-\sqrt{z} }=\frac{e^{z}(1+\sqrt{z} )}{1-z}
.\]
Then $z=1$ is clearly a simple pole, so the residue is 
\[
Res[\frac{e^{z}(1+\sqrt{z} )}{1-z}; 1]=-2e
.\]

\medskip


\noindent
3. (20)  Evaluate the following integrals by Cauchy residue Theorem

(a) $ \int_{|z|=3} \frac{ e^z}{ (z-1)^2 z^3} $

Calculating residue: 
\[
Res[f(z);1]=\frac{d}{dx}\left( \frac{e^{z}}{z^3} \right)=\frac{z^3e^{z}-3z^2e^{z}}{z^{6}}\bigg|_{z=1}=-2e
.\]
\[
Res[f(z);0]=2\frac{d^2}{dx^2}\left( \frac{e^{z}}{(z-1)^2} \right)=\frac{1}{2} \frac{d}{dz}\left( \frac{(z-1)^2e^{z}-2(z-1)e^{z}}{(z-1)^{4}} \right)
.\]
\[
=\frac{1}{2}\left( \frac{\left( 2(z-1)e^{z}+(z-1)^2e^{z}-2e^{z}-2(z-1)e^{z} \right)(z-1)^{4}-\left( (z-1)^2e^{z}-2(z-1)e^{z} \right)4(z-1)^{3} }{(z-1)^{8}} \right) \bigg|_{z=0}
.\]
\[
=\frac{1}{2}\left( \left( -2+1-2+2 \right)+4\left( 1+2 \right)  \right) =\frac{11}{2}
.\]
\[
\implies \int_{|z|=3} \frac{ e^z}{ (z-1)^2 z^3}=(\frac{11}{2}-2e)2\pi i
.\]

(b) $\int_{|z|=1} \frac{1}{z^2 \sin z} dz$

The one singularity is $z=0$. 
\[
\frac{1}{z^2\sin z}=\frac{1}{z^3}\frac{1}{1-z^2 /6+O(z^{4})}=\frac{1}{z^3}\left( 1+\left(\frac{z^2}{6}-O(z^{4} )\right)+\ldots\right)\implies Res[\frac{1}{z^2\sin z};0]=\frac{1}{6}
.\]
\[
\implies \int_{|z|=1} \frac{1}{z^2 \sin z} dz=\frac{\pi i}{3}
.\]

(c) $ \int_{|z|=1} e^{\frac{1}{z}} \cos (z) dz $

Calculating residue: 
\[
e^{\frac{1}{z}}\cos(z)=\left( 1+\frac{1}{z}+\frac{1}{2z^2}+\ldots \right) \left( 1-\frac{z^2}{2}+\frac{z^{4}}{24} -\ldots\right) \implies Res[e^{\frac{1}{z}}\cos(z);0]=\sum_{n=1} \frac{1}{n ((n-1)!)^2}
.\]
\[
\implies \int_{|z|=1} e^{\frac{1}{z}} \cos (z) dz=2\pi i \sum_{n=1} \frac{(-1)^{n+1}}{n ((n-1)!)^2}
.\]

(d) $\int_{|z|=1} \frac{e^z}{\sin^3 z} dz $

Computing residue: 
\[
\frac{e^{z}}{\sin^3z}=\frac{1+z+z /2}{\left( z-z^3 /6+\ldots \right)^3 }=\frac{1}{z^3}\left( 1+z+z^2 /2 \right)\left( 1-3\left( 1-z^2 /6+\ldots \right) +\ldots \right)
.\]
\[
\implies Res[\frac{e^{z}}{\sin^3 z};0]=\frac{1}{2}+\frac{3}{3!}=1\implies\int_{|z|=1} \frac{e^z}{\sin^3 z} dz=2\pi i
.\]

\medskip

\noindent
4. Computing the following integrals

(a) $\int_0^{\pi} \frac{1}{ 1+\sin^2 \theta} d \theta $

Let $z=e^{i\theta}\implies dz=ie^{i\theta}d\theta$. 
\[
\int_0^\pi \frac{1}{1+\sin^2\theta}d\theta=\frac{1}{2}\int_{|z|=1}\frac{-iz^{-1}}{1-\left( z-z^{-1} \right)^2 /4}dz=\int_{|z|=1}\frac{-2iz^{-1}}{6-z^2-z^{-2}}dz
.\]
\[
=-8\pi Res\left[\frac{z}{z^{4}-6z^2+1};\sqrt{2}-1 \right]=\frac{\pi}{\sqrt{2} }
.\]

(b) $ \int_0^{2\pi} \frac{ \sin^2 \theta}{ 3+\cos \theta} d \theta $

Let $z=e^{i\theta}\implies dz=ie^{i\theta}d\theta$. 
\[
\int_0^{2\pi}\frac{\sin^2\theta}{3+\cos\theta}d\theta=\int_{|z|=1}\frac{z^2-2+z^{-2}}{-4iz(3+(z+z^{-1}) /2)}dz=\frac{i}{2}\int_{|z|=1}\frac{z^4-2z^2+1}{z^2(z^2+6z+1)}dz
.\]
\[
=-\pi Res\left[ \frac{z^4-2z^2+1}{z^2(z^2+6z+1)};0\right]-\pi Res\left[ \frac{z^4-2z^2+1}{z^2(z^2+6z+1)};2\sqrt{2} -3\right]=\left( 6-4\sqrt{2}  \right) \pi
.\]


\medskip


\noindent
5. (30) Using contour integrals to  compute the following integrals

(a) $\int_0^\infty \frac{ x^2}{ (x^2+4)^2} dx$

Taking the contour to be a disk of radius $R$ in the upper half plane as $R\to \infty$ and noting that the function is even: 

\[
2\int_{0}^\infty \frac{x^2}{(x^2+4)^2}dx=2\pi iRes[\frac{z^2}{(z-2i)^2(z+2i)^2};2i]=2\pi i \frac{d}{dz} \frac{z^2}{(z+2i)^2}\bigg|_{z=2i}
.\]
\[
=2\pi i \frac{2z(z+2i)^2-2z^2(z+2i)}{(z+2i)^{4}}=-2\pi i\frac{i}{8}
.\]
\[
\implies \int_{0}^\infty \frac{x^2}{(x^2+4)^2}dx=\frac{\pi}{8}
.\]

(b) $ \int_0^\infty \frac{1}{ x^4+x^2+1} dx $

Same integration contour as last time with the outside going to zero, and the function is again even: 
\[
2\int_0^\infty \frac{1}{ x^4+x^2+1} dx=2\pi i Res[\frac{1}{ z^4+z^2+1};e^{\frac{\pi}{3}i}]+2\pi i Res[\frac{1}{ z^4+z^2+1};e^{\frac{2\pi}{3}i}]
.\]
\[
=\frac{2\pi i}{2\left( e^{i\frac{\pi}{3}}-2 \right) }-\frac{2\pi i}{2\left(2+e^{i \frac{2\pi}{3}}\right)}=\frac{\pi}{3\sqrt{3} }
.\]

(c) $\int_0^\infty \frac{1}{ x^3+1} dx $.

Let the contour be the wedge with angle $\frac{2\pi}{3}$. Then the boundary term goes to zero and the integral is 
\[
\int_0^{\infty}\frac{1}{x^3+1}dx-\int_0^\infty \frac{e^{\frac{2\pi}{3}i}}{\left( r e^{\frac{2\pi}{3}i} \right)^3+1 }dr=\left( 1-e^{\frac{2\pi i}{3}} \right) \int_0^{\infty}\frac{1}{x^3+1}dx=2\pi iRes[\frac{1}{z^3+1};e^{i\frac{\pi}{3}}]=2\pi i \frac{1}{3\left( e^{i \frac{2\pi}{3}} \right) }
.\]
\[
\implies \int_0^{\frac{1}{x^3}+1}dx=\frac{2\pi}{3\sqrt{3} }
.\]

(d) $ \int_0^\infty \frac{\cos x}{ x^4 +1} dx $

Expand the given integral into the $z$ plane on the upper half disk of radius $R$ as $R\to \infty$. 

\[
2\int_0^\infty \frac{\cos x}{x^{4}+1}dx=Re\left( 2\pi iRes\left[\frac{e^{iz}}{z^{4}+1};e^{i\frac{\pi}{4}}\right]+2\pi iRes\left[\frac{e^{iz}}{z^{4}+1};e^{i\frac{3\pi}{4}}\right]\right)
.\]
\[
=\frac{\pi}{2} Re\left( ie^{i\frac{\pi}{4}-(-1)^{3 /4}}+ie^{i\frac{3\pi}{4}-(-1)^{1 /4}} \right) 
.\]
Computing this numerically, since it doesn't simplify nicely algebraically: 
\[
 \int_0^\infty \frac{\cos x}{x^{4}+1}dx\approx 0.772138 
.\]

(e) $\int_{-\infty}^\infty \frac{ \sin x}{ x^2+2x+2} dx $

Taking the same contour as the previous part: 
\[
\int_{-\infty}^\infty \frac{ \sin x}{ x^2+2x+2} dx=Im\left( 2\pi i Res\left[\frac{e^{z}}{z^2+2z+2};-1+i\right] \right) =Im\left( \pi e^{-1-i} \right) =\frac{-\pi\sin(1)}{e}\approx -0.972551
.\]


\end{document}
