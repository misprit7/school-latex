\documentclass[letterpaper, reqno,11pt]{article}
\usepackage[margin=1.0in]{geometry}
\usepackage{color,latexsym,amsmath,amssymb,graphicx, float}
\usepackage{hyperref}

\hypersetup{
colorlinks=true,
linkcolor=magenta,
filecolor=magenta,
urlcolor=cyan,
}

\graphicspath{ {images/} }

\begin{document}
\pagenumbering{arabic}
\title{MATH 305 Homework 8}
\date{21/03/22}
\author{Xander Naumenko}
\maketitle

\noindent
1. Use Rouche's theorem to count the number of zeroes for $ p(z)= 4z^5+z^2+2z-1 $ in $|z|\leq 1$.

Let $g(z)=4z^{5}$. Then for $\left| z \right| =1$, we have 
\[
\left| p(z)-g(z) \right| =\left| z^2+2z-1 \right|=\left| z+2-z^{-1} \right|=\left| 2+2i\sin\theta \right|   \leq 2\sqrt{2} <4\leq |4z^{5}|
.\]
Thus by Rouch\'e's theorem there are 5 zeros inside $|z|\leq 1$. 

\medskip



\noindent
2. Use Rouche's theorem to count the number of zeroes for $ p(z)= z^5 + 7 z^2+2 $ in $ 1\leq |z| \leq 2 $.

Let $g(z)=z^{5}$. Then for $|z|=2$, we have 
\[
\left| p(z)-g(z) \right| =|7z^2+2|\leq 26<32=|z^{5}|
.\]
Let $h(z)=7z^2$. Then for $|z|=1$, we have 
 \[
\left| p(z)-h(z) \right| =|z^{5}+2|\leq 3 <7=|7z^{2}|
.\]
Thus by Rouch\'e's theorem, there are $5$ zeros within $|z|\leq 2$ and $2$ in $|z|\leq 1$, which means that there are 3 zeros with $1\leq|z|\leq 2$. 


\medskip

\noindent
3. Use Rouche's theorem to count the number of zeroes for $ f(z)= z^2-4 + 3 e^{-z}$ on the right half plane $ \{ Re (z)>0\}$.

Let $g(z)=z^2-4$, and consider the right half circle with radius $R$ and origin $z=0$ as $R\to\infty$. Then we have for $Re(z)\geq 0$ and $z=R$: 
\[
\left| f(z)-g(z) \right| =\left| 3e^{-z} \right| =3 e^{-Re(z)}\leq 3<\left| z^2-4 \right|<R^2\text{ as }R\to\infty
.\]
Instead for $0\leq |z|\leq R, Re(z)=0$, we have: 
 \[
\left| f(z)-g(z) \right| =\left| 3e^{-z} \right| =3<4\leq \left| z^2-4 \right|
.\]
Since the equality holds true on the contour as $R\to\infty$, we have that the number of zeros of $g$ is the same as the number of zeros of $f$. In this case $g$ has 1 zero on the right half plane, so $f$ has one zero on the right half plane. 

\medskip

\noindent
4. Use Nyquist criterion to find the number of zeroes of $ p(z)= z^3+2z^2 +4$ in the right half plane $ \{ Re(z)>0\}$.

Using the Nyquist criterion: 
\[
N=\frac{1}{2\pi}\left( 3\pi+2[arg p]_{\Gamma_{I_+}} \right) 
.\]
Note that at $p(iy)=-iy^3-2y^2+4=(4-2y^2)-iy^3=p_r(y)+ip_i(y)$

\begin{table}[htpb]
    \centering
    \label{tab:q4}
    \begin{tabular}{c|c|c}
        $\Gamma$&$p_r$&$p_i$\\
        \hline
        $\infty$&$<0$&$<0$\\
        $2$ &$0$& $<0$\\
        $0$&$4$& $0$
    \end{tabular}
\end{table}
Therefore 
\[
2[arg p]_{\Gamma_{I_+}}=\frac{\pi}{2}
.\]
Therefore by the Nyquist criteria listed above we have $N=2$


\medskip

\noindent
5. Use Nyquist criterion to find the number of zeroes of $ p(z)= z^3+2z^2+4z +2$ in the right half plane $ \{ Re(z)>0\}$.

Expanding for $z=iy$: $p(iy)=-iy^3-2y^2+4iy+2=2-2y^2+i(-y^3+4y)=p_r(y)+p_i(y)$. To calculate the argument for the imaginary line construct a table: 

\begin{table}[htpb]
    \centering
    \label{tab:q4}
    \begin{tabular}{c|c|c}
        $\Gamma$&$p_r$&$p_i$\\
        \hline
        $\infty$&$<0$&$<0$\\
        $2$ &$<0$& $0$\\
        $1$ &$0$& $>0$\\
        $0$&$2$& $0$
    \end{tabular}
\end{table}

Thus we have: 
\[
2[arg p]_{\Gamma_{I_+}}=\frac{-3\pi}{2}
.\]
And by Nyquist's criterion: 
\[
N=\frac{1}{2\pi}\left( 3\pi+2[arg p]_{\Gamma_{I_+}} \right)=0
.\]

\medskip

\noindent
6. (20pts)  (a) Use Nyquist criterion to show that there are no zeroes of $ p(z)= z^3+z^2+4z+1$ in  $\{ Re(z) \geq 0\}$.


Expanding for $z=iy$: $p(iy)=-iy^3-y^2+4iy+1=1-y^2+i(-y^3+4y)=p_r(y)+p_i(y)$. To calculate the argument for the imaginary line construct a table: 

\begin{table}[htpb]
    \centering
    \label{tab:q4}
    \begin{tabular}{c|c|c}
        $\Gamma$&$p_r$&$p_i$\\
        \hline
        $\infty$&$<0$&$<0$\\
        $2$ &$<0$& $0$\\
        $1$ &$0$& $>0$\\
        $0$&$2$& $0$
    \end{tabular}
\end{table}

Thus we have: 
\[
2[arg p]_{\Gamma_{I_+}}=\frac{-3\pi}{2}
.\]
And by Nyquist's criterion: 
\[
N=\frac{1}{2\pi}\left( 3\pi+2[arg p]_{\Gamma_{I_+}} \right)=0
.\]

\medskip

\noindent
(b). Show that all  the solutions $ y= y(t)$ to
$$ y^{'''}+ y^{''} + 4 y^{'} +y=0$$
must approach to zero as $ t\to +\infty$.

Taking the laplace transform of the differential equation: 
\[
\mathcal L(y'''+ y''+ 4 y' +y)=Y(s)\left( s^3+s^2+4s+1 \right)=Q(s)\implies Y(s)=\frac{Q(s)}{s^3+s^2+4s+1}
.\]

We just proved in part a that all zeros of the denominator lie in the right half plane, which means all solutions trend to $0$ as $t\to 0$. 

\medskip

\noindent
7. (20pts) Find the Laurent series for the function $ \frac{z}{ (z+1)(z-2)}$ in each of the following domains

(a) $ |z|<1$

Partial fraction decomposition: 
\[
\frac{z}{(z+1)(z-2)}=\frac{1}{3(z+1)}+\frac{2}{3(z-2)}=\frac{1}{3}\left( \frac{1}{1+z}-\frac{1}{1-\frac{z}{2}} \right) =\frac{1}{3}\sum_{n=0}^\infty \left( z^{n}(-1)^{n}-\frac{z^{n}}{2^{n}} \right) 
.\]

(b) $ |z|>2$

Expanding: 
\[
\frac{z}{(z+1)(z-2)}=\frac{1}{3(z+1)}+\frac{2}{3(z-2)}=\frac{1}{3z}\left( \frac{1}{1+\frac{1}{z}}+\frac{2}{1-\frac{2}{z}} \right) =\frac{1}{3z}\sum_{n=0}^\infty \left(\frac{(-1)^{n}}{z^{n}}+\frac{2^{n+1}}{z^{n}}\right)
.\]
\[
=\frac{1}{3}\sum_{n=1}^\infty \left(-\frac{(-1)^{n}}{z^{n}}+\frac{2^{n}}{z^{n}}\right)
.\]

\medskip

\noindent
8. Find the first three terms of Laurent series for $\frac{z}{ \mbox{Log} (z)}$   in $|z-1| <1$, where $ Log (z)$ is the principal branch.

First note that we have that the Taylor series of $\log(z)$ is: 
\[
\log(1-z+1)=\sum_{n=1}^{\infty}\frac{(z-1)^{n}(-1)^{n}}{n}
.\]
Thus we have:
\[
\frac{z}{\log(z)}=\frac{(z-1)+1}{z-1}\frac{1}{1-((z-1) /2-(z-1)^2 /3-\ldots)}
.\]
\[
=\left( 1+\frac{1}{z-1} \right) \left( 1+((z-1) /2-(z-1)^2 /3-\ldots)+((z-1) /2-(z-1)^2 /3-\ldots)^2+\ldots \right) 
.\]
\[
=\frac{1}{z-1}+\frac{3}{2}+\frac{5(z-1)}{12}
.\]

 

\end{document}
