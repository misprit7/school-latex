%
%       Template file generated by TeXmenu v2.2, June 1992, patchlevel 431.
%
\documentstyle[12pt]{article}
\textwidth=6.8in
\textheight=8.8in
\oddsidemargin=-0.1in
\evensidemargin=-0.1in
%\baselineskip=16pt


\begin{document}





\noindent
{\bf MATH305-201-2021/2022 Homework Assignment 4 (Due Date: Feb. 7, 2022) }

\vskip 0.5cm

10pts each

\medskip



\noindent
1. Find a conformal mapping from the following set onto the upper half plane $ S^{'}=\{(u,v) \ | \ v>0\}$:

(a) $ S= \{  x>0, -\frac{\pi}{2}<y<\frac{\pi}{2} \}$; (b) $ S= \{ -1<x<3, y>0\}$

 Hint: use the linear map $ az+b$ and $\sin (z)$.



\noindent
2. Evaluate the following

(a) $log (i)$; (b) $Log (\sqrt{3}-i)$; (c)  $ log ( e^{1+i})$; (d) $e^{log (1+i)}$


\medskip

\noindent
3. Find all values of

(a) $ e^z= -1-i$;   (b) Principal Values of $ (1+i)^{i}$; (d) $i^{\frac{1}{3}}$

\medskip

\noindent
4. Solve the following equations

(a) $ Log (z^2-1)= \frac{i\pi}{2}$; (b) $ e^{2z}+ e^z+1=0$; (c) $ z^{\frac{1}{2}} +1-i=0$ (here $z^{\frac{1}{2}}$ denotes the principal branch)


\medskip

\noindent
5. Determine the domain of analyticity (branch cut) of

(a) $ Log (1+z^2)$; (b) $ Log (\frac{1-z}{1+z})$


\medskip

\noindent
6. Which of the followings are true statements? For the ones that are false find a counterexample

(a) $ e^{ log (z)} = z $; (b) $ e^{ Log (z)}=z$; (c) $ Log (e^z)=z$; (d) $ log (e^z)=z$;  (e) $ log (z_1 z_2)=log z_1+ log z_2 $; (f) $ log (z)=- log (\frac{1}{z})$; (g) $ log (z^{\frac{1}{2}})=\frac{1}{2} log (z)$

\medskip


\noindent
7. Find a branch cut of $ log (z-1)$ that is analytic at all points in the plane except those on the following rays.

(a) $ \{ x\leq 1, y=0 \}$; (b) $ \{ x \geq 1, y=0 \}$; (c) $\{ x=1, y \geq 0 \}$


\medskip




\medskip


\noindent
8. Find a branch cut for $\sqrt{z (z-1)}$ that is analytic in $ C \backslash [0, 1]$ and takes value $ \sqrt{2} $ at $z=2$.

\medskip


\noindent
9. Determine a branch of $ log (z^2+2z+2)$ that is analytic at $z=-1$ and takes value $ 0$ at $z=-1$, and find its derivative there.


\medskip

\noindent
10. Determine a branch of $log (1+z^2)$ that is analytic at $z=0$ and takes the value $2 \pi i$ there.





\end{document}






