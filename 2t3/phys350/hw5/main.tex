\documentclass[letterpaper, reqno,11pt]{article}
\usepackage[margin=1.0in]{geometry}
\usepackage{color,latexsym,amsmath,amssymb,graphicx, float}
\usepackage{hyperref}

\hypersetup{
colorlinks=true,
linkcolor=magenta,
filecolor=magenta,
urlcolor=cyan,
}

\graphicspath{ {images/} }

\begin{document}
\pagenumbering{arabic}
\title{PHYS 350 Homework 5}
\date{28/03/22}
\author{Xander Naumenko}
\maketitle

{\noindent\bf Question 1.} For this problem $s=1$, and choose $q=\theta$. We tie the IRF to the bottom left corner of the wall. Then for the center of mass of the rod, we have: 
\[
x_{CM}=\frac{1}{2}\cos\theta, \dot x_{CM}=-\frac{1}{2}\dot\theta\sin\theta
.\]
\[
y_{CM}=\frac{1}{2}\sin\theta, \dot y_{CM}=\frac{1}{2}\dot\theta\cos\theta
.\]
For the rod's kinetic energy, note that the moment of inertia of the rod is $I=\frac{ML^2}{12}$. Then we have: 
\[
\mathcal L=T-U=\frac{1}{2}mv_{CM}^2+\frac{1}{2}\vec\Omega \hat{I}\vec\Omega-\frac{L}{2}mg\sin\theta=\frac{1}{6}mL^2\dot\theta^2-\frac{L}{2}mg\sin\theta
.\]
Because there's no time dependence we can find the energy in the system: 
\[
E=\frac{d\mathcal L}{d\dot\theta}\dot \theta-\mathcal L=\frac{1}{6}mL^2\dot\theta^2+\frac{L}{2}mg\sin\theta=\frac{L}{2}mg\sin\theta_0
.\]
\[
\implies \dot\theta^2=\frac{3g(\sin\theta_0-\sin\theta)}{L}
.\]
\[
\implies T=\int_0^T dt=\int^{\theta_0}_0 \sqrt{\frac{L}{3g(\sin\theta_0-\sin\theta)}}d\theta
.\]

{\noindent\bf Question 2.} For this problem $s=1$, and choose $ q=\theta$ where $\theta$ is the angle between the vertical and the line marked $d$ on the diagram. First we calculate the moment of inertia by noting that with the parallel axis theorem, two times the moment of inertia displaced by distance $d$ to the center should be equal to the moment of inertia of the semicircle, i.e.
\[
2(I_{CM}+Md^2)=\frac{2}{2}MR^2\implies I_{CM}=\frac{1}{2}MR^2-Md^2
.\]
Next we consider the kinetic energy at the point $O$ at the circle touches the ground, with $r'$ being the distance from the CM to $O$. 
\[
r'^2=d^2+R^2-2dR\cos\theta
.\]
\[
I_O=I_{CM}+r'^2M=\frac{3}{2}MR^2-2dRM\cos\theta
.\]
\[
\mathcal L=T-U=\left( \frac{3}{4}R-dM\cos\theta \right) R\dot\theta^2-Mgd\cos\theta
.\]
Finding stable points (with restrictions on $\theta_0$ given the shape): 
\[
\frac{dU}{d\theta}=Mgd\sin\theta\implies\theta=0
.\]
\[
\frac{d^2U}{d\theta^2}=Mg\cos\theta\bigg|_{\theta=0}\geq 0\implies \theta=0\text{ is stable}
.\]
Estimating the Lagrangian: 
\[
\mathcal L\approx \left( \frac{3}{4}R-d \right) RM\ddot\theta^2-\frac{1}{2}Mgd\theta^2
.\]
Euler Lagrange: 
\[
    \left( \frac{3}{2}R-2d \right) RM\ddot\theta=-Mgd\theta
.\]
\[
\implies \omega=\sqrt{\frac{2gd}{3R^2-4dR}} 
.\]

{\noindent\bf Question 3.} Here $s=1$, choose $q=x$ where $x$ is the distance from the bottom of the hill to the center of the cart. Tie our IRF to the bottom of the slope. The only term that is rotating is the wheels, whose kinetic energy is: 
\[
T_{wheels}=2\cdot\frac{1}{2}I\omega^2=\frac{1}{2}m_wR^2\left( \frac{\dot x}{R} \right)^2=\frac{1}{2}m_w \dot x^2
.\]
Let $M_T=2m_w+m_F+M$ be the total mass of cart and rider. All of the rest of the bodies just act as point masses (since they aren't rotating), so we get the Lagrangian to be: 
\[
\mathcal L=\frac{1}{2}M_T\dot x^2+\frac{1}{2}m_w\dot x^2-M_T gx\sin\alpha
.\]
Euler Lagrange: 
\[
    \left( M_T+m_w \right) \ddot x=-M_T g\sin\alpha\implies \ddot x=\frac{-M_T g\sin\alpha}{M_T+m_w}
.\]
To maximize the acceleration, you want the fraction $\frac{M_T}{M_T+m_w}$ to be as large as possible. This means that you could either make your wheel mass smaller, or you could make the rest of your cart heavier. However since $m_w> 0$ your acceleration will always be worse than the perfect sliding case. 

{\noindent\bf Question 4.} We choose the coordinates as follows: 
\[
x_{CM}=\frac{2x_2+x_1}{3}, x_{rel}=x_2-x_1
.\]
Then note that we have 
\[
x_2=x_{CM}+\frac{1}{3}x_{rel}, x_1=x_{CM}-\frac{2}{3}x_{rel}
.\]
Putting these together: 
\[
U=\frac{1}{2}k\left( x_{rel}-l_0 \right)^2\approx\frac{1}{2}kx_{rel}^2-kx_{rel}l_0 
.\]
\[
T=\frac{3}{2}m \dot x_{CM}^2+\frac{1}{2}\frac{2m^2}{2m}\dot x_{rel}^2+\frac{1}{4}mR^2 \frac{\dot x_1^2}{R^2}+\frac{1}{2}mR^2 \frac{\dot x_2^2}{R^2}=\frac{3}{2}m\dot x_{CM}^2+\frac{1}{2}m\dot x_{rel}^2+\frac{3}{4}m\dot x_{CM}^2+\frac{1}{2}m\dot x_{rel}^2
.\]
\[
=\frac{9}{4}mx_{CM}^2+\frac{1}{2}m\dot x_{rel}^2
.\]
\[
\mathcal L=\frac{9}{4}m\dot x_{CM}^2+\frac{1}{6}m\dot x_{rel}^2-\frac{1}{2}kx_{rel}^2+kx_{rel}l_0
.\]
Euler Lagrange: 
\[
\frac{9}{2}m\ddot x_{CM}=0\implies x_{CM}=-\frac{5}{3}v_0 t
.\]
\[
m\ddot x_{rel}=kl_0-kx_{rel}\implies \ddot x_{rel}=\frac{kl_0}{m}-\frac{k}{m}x_{rel}
.\]
Let $\omega=\sqrt{\frac{k}{m}} $. Then we get: 
\[
x_{rel}=l_0-\frac{v_0}{\omega}\sin(\omega t)
.\]
\[
x_1=-\frac{5}{3}v_0 t-\frac{2}{3}\left( l_0-\frac{v_0}{\omega}\sin(\omega t) \right) 
.\]
\[
x_2=-\frac{5}{3}v_0 t+\frac{1}{3}\left( l_0-\frac{v_0}{\omega}\sin(\omega t) \right) 
.\]

Note that here the coordinates weren't specified, so we assume that the center of mass starts at $0$. 

{\noindent\bf Question 5.} There is only one degree of freedom, $q=l$. Then the Lagrangian is: 
\[
\mathcal L=T-U=\frac{1}{2}m_1\dot l^2+\frac{1}{2}m\dot l^2+\frac{1}{2}I\omega^2+m_1gl\sin\alpha+\frac{1}{2}\frac{m}{L}l^2g\sin\alpha
.\]
\[
\mathcal L=\frac{1}{2}\left( m_1+m+\frac{1}{2}M \right)\dot l^2+m_1gl\sin\alpha+\frac{1}{2}\frac{m}{L}l^2g\sin\alpha
.\]
Taking energy: 
\[
E=\frac{1}{2}\left( m_1+m+\frac{1}{2}M \right)\dot l^2-m_1gl\sin\alpha-\frac{1}{2}\frac{m}{L}l^2g\sin\alpha=0
.\]
\[
\implies \dot l=\sqrt{\frac{2m_1 g\sin\alpha}{m_1+m+\frac{1}{2}M}\left( l+\frac{1}{2}\frac{m}{m_1L}l^2 \right) } 
.\]
\[
t=\int_0^t dt'=\int_0^l \sqrt{\frac{m_1+m+\frac{1}{2}M}{2m_1 g\sin\alpha}\frac1{l'+\frac{1}{2}\frac{m}{m_1L}l'^2} }dl'
.\]
Let $a=\frac{1}{2}\frac{m}{m_1L}$ and $b=\sqrt{\frac{m_1+m+\frac{1}{2}M}{2m_1g\sin\alpha}} $. Then the integral becomes: 
\[
t=\int_0^l \frac{b}{\sqrt{l'+al'^2} }dl'=\frac{2b}{\sqrt{a} }\text{arcsinh}\left( \sqrt{al}  \right) \implies l(t)=\frac{1}{a}\sinh^2\left( \frac{t\sqrt{a}}{2b} \right) 
.\]
\[
\implies l(t)=\frac{2m_1L}{m}\sinh^2\left( t \frac{1}{2}\sqrt{\frac{gm\sin\alpha}{(m_1+m+\frac{1}{2}M)L}}  \right) 
.\]

\end{document}
