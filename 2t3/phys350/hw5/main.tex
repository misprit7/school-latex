\documentclass[letterpaper, reqno,11pt]{article}
\usepackage[margin=1.0in]{geometry}
\usepackage{color,latexsym,amsmath,amssymb,graphicx, float}
\usepackage{hyperref}

\hypersetup{
colorlinks=true,
linkcolor=magenta,
filecolor=magenta,
urlcolor=cyan,
}

\graphicspath{ {images/} }

\begin{document}
\pagenumbering{arabic}
\title{PHYS 350 Homework 5}
\date{28/03/22}
\author{Xander Naumenko}
\maketitle

{\noindent\bf Question 1.} For this problem $s=1$, and choose $q=\theta$. We tie the IRF to the bottom left corner of the wall. Then for the center of mass of the rod, we have: 
\[
x_{CM}=\frac{1}{2}\cos\theta, \dot x_{CM}=-\frac{1}{2}\dot\theta\sin\theta
.\]
\[
y_{CM}=\frac{1}{2}\sin\theta, \dot y_{CM}=\frac{1}{2}\dot\theta\cos\theta
.\]
For the rod's kinetic energy, note that the moment of inertia of the rod is $I=\frac{ML^2}{12}$. Then we have: 
\[
\mathcal L=T-U=\frac{1}{2}mv_{CM}^2+\frac{1}{2}\vec\Omega \hat{I}\vec\Omega-\frac{L}{2}mg\sin\theta=\frac{1}{6}mL^2\dot\theta^2-\frac{L}{2}mg\sin\theta
.\]
Because there's no time dependence we can find the energy in the system: 
\[
E=\frac{d\mathcal L}{d\dot\theta}\dot \theta-\mathcal L=\frac{1}{6}mL^2\dot\theta^2+\frac{L}{2}mg\sin\theta=\frac{L}{2}mg\sin\theta_0
.\]
\[
\implies \dot\theta^2=\frac{3g(\sin\theta_0-\sin\theta)}{L}
.\]
\[
\implies T=\int_0^T dt=\int_{\theta_0}^0 \sqrt{\frac{L}{3g(\sin\theta_0-\sin\theta)}}d\theta
.\]

{\noindent\bf Question 2.} For this problem $s=1$, and choose $ q=\theta$ where $\theta$ is the angle between the vertical and the line marked $d$ on the diagram. First we calculate the moment of inertia by noting that with the parallel axis theorem, two times the moment of inertia displaced by distance $d$ to the center should be equal to the moment of inertia of the semicircle, i.e.
\[
2(I_{CM}+Md^2)=\frac{1}{2}MR^2\implies I_{CM}=\frac{1}{4}MR^2-Md^2
.\]
Next we consider the kinetic energy at the point $O$ at the circle touches the ground, with $r'$ being the distance from the CM to $O$. 
\[
r'^2=d^2+R^2-2dR\cos\theta
.\]
\[
I_O=I_{CM}+r'^2M=\frac{5}{4}MR^2-2dRM\cos\theta
.\]
\[
\mathcal L=T-U=\left( \frac{5}{8}R-dM\cos\theta \right) R\dot\theta^2-Mgd\cos\theta
.\]

\end{document}
