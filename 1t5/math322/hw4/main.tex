\documentclass[letterpaper, reqno,11pt]{article}
\usepackage[margin=1.0in]{geometry}
\usepackage{color,latexsym,amsmath,amssymb,graphicx,float,listings,tikz}
\usepackage{hyperref}

\hypersetup{
colorlinks=true,
linkcolor=magenta,
filecolor=magenta,
urlcolor=cyan,
}

\graphicspath{ {images/} }

\begin{document}
\pagenumbering{arabic}
\title{Math 322 Homework 4}
\date{03/10/23}
\author{Xander Naumenko}
\maketitle

{\medskip\noindent\bf Question 2.} Let $a\in M$. Since $M=\langle S\rangle$, we can write $a$ as $a=s_1\cdot s_2\cdots s_k, s_i\in S\forall i=1,2\ldots,k$. Then we have $a^{-1}=s_k^{-1}\cdots s_1^{-1}$, so each element in $M$ is invertible and thus $M$ is a group.

{\medskip\noindent\bf Question 5.} Let $S=\{q_1,q_2,\ldots, q_n\}\subset \mathbb{Q}$. We can write each $q_i$ as $q_i=\frac{a_i}{b_i}$ for some $a_i\in \mathbb{Z}, b_i\in \mathbb{N}, \gcd(a_i,b_i)=1$. Define $q=\frac{1}{\text{lcm}(b_1,b_2,\ldots, b_n)}$. Then for each $q_i$, we can write $q_i=qm$ for some $m\in \mathbb{Z}$. Thus $\langle S\rangle=\langle q\rangle$, i.e. $S$ is cyclic.

For the second part, let $\phi:\mathbb{Q}\times \mathbb{Q}\to \mathbb{Q}$ be a map, we will show by contradiction that $\phi$ can't be an isomorphism, so for now assume that it is one. Let $G$ be the group generated by $(1,0)$ and $(0,1)$. By the result from the first part and the fact that $\phi$ is supposedly an isomorphism we have that $G= \langle\left( q_1,q_2 \right)\rangle$ with at least one of $q_1,q_2\neq 0$. However then $(1,0)=a(q_1,q_2),(0,1)=b(q_1,q_2)$ for some $a,b$ since $(1,0)\in G, (0,1)\in G$. However this implies that $a=0,b=0\implies q_1=0$ or $q_2=0$ which is clearly a contradiction, so $\mathbb{Q}$ isn't isomorphic to the direct product of itself.

{\medskip\noindent\bf Question 6.} By way of contradiction let $x\in \langle a\rangle$ and $x\in\langle h\rangle$ with $x\neq 1$. Then since $x$ is in a cyclic subgroup generated by $a$ we can write $x=a^{k}$ for some $k<m$, and it must be that $x^{m}=x^{n}=1$. Without loss of generality assume $m>n$, then $x^{m-n}=a^{k(m-n)}=1\implies m|k(m-n)$. However $k<m$ and $m-n$ shares no factors with $m$, so this is clearly impossible and it must instead be that $\langle a\rangle \cap\langle b\rangle =1$.

For the second part, note firstly that clearly $\langle ab\rangle\subset\langle a,b\rangle$ since any $ab^{k}=a^{k}b^{k}$. For the other direction, let $x=a^{k}b^{l}\in \langle a,b\rangle$. Let $c$ be a solution to the set of modular equations $c\equiv 0\mod n,c\equiv k\mod m$ and similarly $d$ be the solution to $d\equiv 0\mod m, d\equiv l\mod n$. Such solutions are guaranteed to exist since $(m,n)=1$. Then $(ab)^{c+d}=a^{c+d}b^{c+d}=a^{c}b^{d}=a^{k}b^{l}$. Thus both sets contain one another and $\langle a,b\rangle=\langle ab\rangle$.

{\medskip\noindent\bf Question 7.} For an element $p\in\langle a\rangle$ with $p=a^{sx+y}$ for some $0\leq x<r,0\leq y<s$ (we can write it this way due to the division algorithm), define $\phi:\langle a\rangle\to\langle a\rangle\times\langle b\rangle$ as $\phi(p)=(a^{x},b^{y})$. Clearly the identity is preserved over this map, so only the preservation of the product is required. Let $p,q\in\langle a\rangle$ with $p=a^{sx_1+y_1}$ and $q=a^{sx_2+y_2}$. Then $\phi(pq)=\phi\left(a^{s(x_1+x_2)+y_1+y_2}\right)=\left( b^{x_1+x_2},c^{y_1+y_2} \right) =\phi(p)\phi(q)$ as required.

We can apply what we just proved iteratively $k$ times to any $o(a)=n=P_1^{\alpha_1}\cdots P_k^{\alpha_k}$ to show that $\langle a\rangle=\langle P_1^{\alpha_1}\rangle\cdots\langle P_k^{\alpha_k}\rangle$. Thus any finite cyclic group is isomorphic to a direct product of cyclic groups of prime power orders.

\end{document}
