\documentclass[letterpaper, reqno,11pt]{article}
\usepackage[margin=1.0in]{geometry}
\usepackage{color,latexsym,amsmath,amssymb,graphicx,float,listings,tikz}
\usepackage{hyperref}

\hypersetup{
colorlinks=true,
linkcolor=magenta,
filecolor=magenta,
urlcolor=cyan,
}

\graphicspath{ {images/} }

\begin{document}
\pagenumbering{arabic}
\title{Math 322 Homework 7}
\date{23/10/23}
\author{Xander Naumenko}
\maketitle

{\medskip\noindent\bf Question 10.} Let $g\in G$. Define $A^{-1}g=\{a^{-1}g: a\in A\}$. Since $|A^{-1}g|=A$, we have $\left| A^{-1}g \right| +B>|G|\implies \exists a\in A$ s.t. $a^{-1}g=b\implies g=ab$. Since this is true of all $g\in G$, we have that $G=AB$.

{\medskip\noindent\bf Question 11.} As the hint suggests, using exercise 13 on page 36 (which we proved in a previous homework), there exists $a\in G$ such that $a^2=1$. Consider $G_L$, the group of left translations, since $G_L<S_{2k}$ is isomorphic to $G$ it suffices to show that there is a subgroup of index 2 in $G_L$. Since $a_L$ has order 2, it is the disjoint union of $k$ transpositions. Let $H$ be the group of all even elements of $G_L$ (it is a group since $1$ is even and the product of two even cycles is even) and let $g_L\in G_L$. If $\text{sg }g_L=1$ then $g_L\in H$. Otherwise if $\text{sg }g_L=-1$ then $g_La_L^{-1}\in H\implies g_L=Ha_L$. Thus $G_L=H\cup Ha_L\implies [G:H]=2$ as required.

{\medskip\noindent\bf Question 2.} Identity:
\[
    (0,0,0)(k,l,m)=(k+0+0,l+0,m+0)=(k,l,m)=(k,l,m)(0,0,0)
.\]
Associativity:
\[
    (k_1,l_1,m_1)\left( (k_2,l_2,m_2)(k_3,l_3,m_3) \right) =(k_1+k_2+l_2m_3+l_1(m_2+m_3), l_1+l_2+l_3,m_1+m_2+m_3)
\]
\[
\left((k_1,l_1,m_1) (k_2,l_2,m_2)\right)(k_3,l_3,m_3)
.\]
Invertibility:
\[
    (k,l,m)(-k+lm,-l,-m)=\left( 0,0,0 \right) =1
.\]
For any $g=(k,l,m)\in G$, we have that for all $c=(t,0,0)\in C$, $gcg^{-1}=(k,l,m)(t,0,0)(-k+lm,-l,-m)=\left( k+t\left( -k+lm \right)-lm,0,0  \right) \in C$, so $C$ is normal. I claim that $\phi:G/C\to Z^{(2)}$ defined as $\phi(k,l,m)=(l,m)$ is bijective. It is well defined, since $C(k,l,m)=C(k',l,m)$. It is injective and onto, since unique choices of $l,m$ uniquely determine the input and output of the function. Thus $G/C\cong \mathbb{Z}^{(2)}$.

{\medskip\noindent\bf Question 4.} Let $G=\langle a\rangle$ with infinite order, and let $\phi$ be an automorphism on $G$. $a$ and $a^{-1}$ are generators of $G$, the only possibilities are that $\phi(a)=a$ or $\phi(a)=a^{-1}$ and hence $\phi(x)=x$ or $\phi(x)=x^{-1}$

Let $G=\langle a\rangle$ with $|G|=6$. For an automorphism $\phi$ on $G$, by the definition of a homomorphism we have that $\phi(a^{k})=\phi(a)^{k}$. Since $\phi(G)=G$, it must be that $|\phi(a)|=6$. Specifically, by theorem 1.3 this implies that $\phi(a)=a^{m}$ for some $m$ such that $(m,6)=1$, i.e. $m=1$ or $5$. Clearly $\phi$ is uniquely determined by the choice of $\phi(a)$, and for any such $m$ we have that $\phi$ is an injective map from $G$ to $G$, so $\phi$ is an automorphism.

Finally for a general finite cyclic group $G$, following the exact same logic as part ii shows that $\phi$ is an automorphism if and only if $\phi(a)=a^{m}$ for some $m$ with $(m,|G|)=1$.

{\medskip\noindent\bf Question 5.} Note that the elements $a=(123)$ and $b=(12)$ generate $S_3$. Powers of each individually generate 4 elements of $S_3$, while $ab=(13)$ and $ba=(23)$. Thus any automorphism $\phi$ is completely determined by $\phi(a)$ and $\phi(b)$ by theorem 1.7. Since there are 2 elements of order $3=|a|$ and 3 elements of order $2=|b|$, there are at most 6 possible automorphisms. Directly trying them all, we see that the possibilities are listed in table \ref{tab:S3}. Since there are at most 6 automorphisms and we've found 6 that work, we're done.
\begin{table}[htpb]
    \centering
    \caption{Possibilities for choices of automorphisms of $S_3$ in question 5.}
    \label{tab:S3}
    \begin{tabular}{|c|c|}
        \hline
        $\phi(a)$&$\phi(b)$\\
        \hline
        $a$&$b$\\
        $a$&$ab$\\
        $a$&$a^2b$\\
        $a^2$&$b$\\
        $a^2$&$ab$\\
        $a^2$&$a^2b$\\
        \hline
    \end{tabular}
\end{table}

{\medskip\noindent\bf Question 8.} For some element $a\in G$ consider the map $\phi: x\to axa^{-1}$. $\phi$ is an automorphism, since $\phi(xy)=axya^{-1}=axa^{-1}aya^{-1}=\phi(x)\phi(y)$. Since $\text{Aut }G=1$, we have that $axa^{-1}=x\implies ax=xa$ for all $x\in G$. Thus $G$ is abelian. Also since $G$ is abelian the map $\psi: x\to x^{-1}$ is an automorphism, since $\psi(xy)=(xy)^{-1}=y^{-1}x^{-1}=x^{-1}y^{-1}=\psi(x)\psi(y)$. Again since $\text{Aut }G=1$, we then have $x=x^{-1}\implies x^2=1$ for all $x\in G$.

To prove the last part, as the hint suggests we will show that there exists unique representation of each element in terms of a fixed set of elements. This property will be shown by induction on subgroups $H_n\leq G$ with $|H_n|=n$. Clearly for $n=1$ the result holds choosing $H=\{1\}$ and $a_1=1$. Assume that $H_n$ has a set of elements $a_1,a_2,\ldots, a_n$ with each element $h\in H$ uniquely represented as $a_1^{k_1}\cdots a_n^{k_n},k_i=0,1$. Clearly these generate $H$ so $H=\langle a_1,\ldots,a_n\rangle$. Choose $a_{n+1}\in G-H$. Since $|a_{n+1}|=2$, $\langle a_{n+1}\rangle\cap H=1$. Also because $G$ is abelian and $|a_{n+1}|=2$ any element $h\in\langle H,a_{n+1}\rangle$ can be written as $ha_{n+1}^{k_{n+1}}$. This representation is unique, since if $ha_{n+1}^{k_{n+1}}=h'a_{n+1}^{k'_{n+1}}\implies h(h')^{-1}=a_{n+1}^{k_{n+1}-k'_{n+1}}\implies h=h',k_{n+1}=k'_{n+1}$ (since $H\cap \langle a_{n+1}\rangle=1$. Thus $\langle H,a_{n+1}\rangle$ is a subgroup with the required property, and so the property holds for all $n$ including $n=|G|$.

Therefore the map $\phi: a_1^{k_1}a_2^{k_2}\cdots a_n^{k_n}\to a_2^{k_1}a_1^{k_2}\cdots a_n^{k_n}$ is an automorphism since it just involves relabeling interchangeable generating elements. Since by assumption $\text{Aut }G=1$, the only way to avoid such a map is if $G=\langle a\rangle$, and since $|a|=2$, $G$ can only either have 1 or 2 elements.

\end{document}
