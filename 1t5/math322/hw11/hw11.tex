\documentclass[letterpaper, reqno,11pt]{article}
\usepackage[margin=1.0in]{geometry}
\usepackage{color,latexsym,amsmath,amssymb,graphicx,float,listings,tikz}
\usepackage{hyperref}

\hypersetup{
colorlinks=true,
linkcolor=magenta,
filecolor=magenta,
urlcolor=cyan,
}

\graphicspath{ {images/} }

\begin{document}
\pagenumbering{arabic}
\title{Math 322 Homework 11}
\date{05/12/23}
\author{Xander Naumenko}
\maketitle

{\medskip\noindent\bf Herstein 2.13.2a.} Consider the map $\phi: G\to D$ defined by $\phi(g)=(g,g)$. $\phi$ is a homomorphism since $\phi(g_1g_2)=(g_1g_2,g_1g_2)=(g_1,g_2)(g_1,g_2)=\phi(g_1)\phi(g_2)$. Also $\ker\phi=\phi^{-1}((1,1))=1$ and $\phi$ is clearly surjective, so $\phi$ shows that $G$ and $D$ are isomorphic.

{\medskip\noindent\bf Question Herstein 2.13.4b.} Both directions:

($\implies$) Suppose $D$ is normal in $T$, and let $g_1,g_2\in G$. Since $D$ is normal we have $(g_2^{-1},g_2^{-1})(g_1,g_1)(g_2,g_2)=(g_1,g_1)\implies g_2^{-1}g_1g_2=g_1\implies g_1g_2=g_2g_1$. Since $g_1,g_2$ were arbitrary thus every element of $G$ commutes, so it is abelian.

($\impliedby$) Suppose $G$ is abelian, and let $(g_1,g_2)\in T,(g,g)\in D$. Then $(g^{-1},g^{-1})(g_1,g_2)(g,g)=(g^{-1}g_1g,g^{-1}g_2g)=(g_1,g_2)$. Thus $D$ is normal in $T$.

{\medskip\noindent\bf Question Herstein 2.13.5.} Let $|G|=\prod_{i=1}^{n}p_i^{\alpha_i}$ and let $P_i$ be a arbitrary Sylow $p_i$-subgroups. Each element in a $P_i$ has order one of $1,p_i,p_i^2,\ldots,p_i^{\alpha_i}$, so other than the identity each of the $P_i$ are pairwise disjoint. Also each $P_i$ is normal since $G$ is abelian. I claim that $G=P_1P_2\cdots P_n$ is the internal direct product of these groups. There are $p_i^{\alpha_i}$ choices for each group, so there are $\prod_{i=1}^{n}p_i^{\alpha_i}=|G|$ elements of the form $g=g_1g_2,\cdots g_n,g_i\in P_i$, I claim that each of these is unique. Suppose $ g_1g_2\cdots g_n=g_1'g_2'\cdots g_n'\implies (g_1g_1'^{-1})^{|G| /p_1^{\alpha_1}}=(g_2'g_2^{-1}\cdots g_n'g_n^{-1})^{|G|/p_1^{\alpha_1}}=1\implies g_1=g_1'$. Repeating this for $2,3,\ldots,n$ gives that this representation of $g$ is unique. Since there are exactly $|G|$ unique elements generated this way, by the definition given on the top of page 106 we have that $G$ is the internal direct product of $P_i$. Then by theorem 2.13.1 it is isomorphic to $P_1\times\ldots\times P_n$.

{\medskip\noindent\bf Question Herstein 2.13.6.} Both directions:

($\implies$) Suppose $A\times B= \langle( a,b )\rangle $ is cyclic. By contradiction assume that $\gcd(m,n)=k>1$, then we have $(a,b)^{\frac{mn}{k}}=\left( \left( a^{m} \right) ^{n /k},\left(b^{n}\right)^{m /k} \right) =(1^{n /k},1^{m /k})=(1,1)$. However this contradicts the assumption that $(a,b)$ was of order $mn$, so it must be that $\gcd(m,n)=1$.

($\impliedby$) Assume that $m$ and $n$ are relatively prime, and let $A=\langle a \rangle $ and $B= \langle b \rangle $. I claim $A\times B= \langle (a,b) \rangle $. Let $k\in \mathbb{N}$ with $(a,b)^{k}=(a^{k},b^{k})=(1,1)$. Since $a^{k}=1$ we have $m|k$ and similarly since $b^{k}=1$ we have $n|k$. $m$ and $n$ are relatively prime so it must be that $mn|k$, implying that the order of $(a,b)=mn$ and thus $A\times B$ is cyclic.

{\medskip\noindent\bf Question 8.} Consider $G=\mathbb{Z}/2\mathbb{Z}\times \mathbb{Z}/2\mathbb{Z}$ using additive notation, let $N_1=\langle(0,1)\rangle$, $N_2= \langle (1,0) \rangle $ and $N_3=(1,1)$. $G$ is abelian so each of these groups is normal, and they each only contain the identity $0$ and their generator so they're clearly disjoint except the identity. Also clearly $G=\{(0,0),(0,1),(1,0),(1,1)\}=N_1N_2N_3$. However $(1,1)$ can be represented either as $(0,1)+(1,0)$ or $(1,1)$, so not every element in $G$ can be uniquely expressed by a product of elements of $N_1,N_2$ and $N_3$. Thus $G$ is not the internal direct product of $N_1,N_2$ and $N_3$.

{\medskip\noindent\bf Question 11.} Let $h\in H_0$ with $h\neq 1$. If $|G|$ has two prime factors $p,q$ then $h$ belongs to both a Sylow $p$-subgroup and Sylow $q$-subgroup, but this is impossible since elements of those groups must have powers that are purely powers of $p$ and $q$ respectively and $h$ can't be both. Thus the order $G$ is $p^{k}$ for some prime $p$ and $k\in \mathbb{N}$.

By Cauchy's theorem there is a subgroup of order $p$ and $H_0$ is contained in it, so we can write $H_0= \langle h \rangle $ where the order of $h$ is $p$. For any $g\in G$ with order $p$ we have $\langle h \rangle \subseteq \langle g \rangle \implies \langle h \rangle =\langle g \rangle $, so this subgroup is unique. Next, I claim that for every $m=1,2,\ldots,k$, there are at most $p^{m}$ elements of order $p^m$. Suppose $g_1,g_2\in G$ both have order $m$, then $\langle h \rangle \subseteq \langle g_1 \rangle $ and $\langle h \rangle \subseteq \langle g_2 \rangle $. Cyclic groups of the same order only intersect nontrivially if they're equal, so $\langle g_1 \rangle =\langle g_2 \rangle$. A group of order $p^{m}$ by definition has exactly $p^{m}$ elements, so the maximum possible number of elements of order $p^{m}$ is $p^{m}$.

Now consider counting the number of elements of each order. The number of elements of order strictly less than $p^{k}$ is, using the above claim (this is, to be clear, a very weak bound but it is sufficient. It ignores the fact that each of these subgroups intersect with all smaller ones),
\[
1+p+p^{2}+\ldots+p^{k-1}=\sum_{i=0}^{k-1}p^{i}=\frac{p^{k}-1}{p-1}\leq p^{k}-1
.\]
However there are $p^{k}$ elements in $G$ so this couldn't have accounted for all of them. Thus there is an element of order $p^{k}$, which implies that $G$ is cyclic.

\end{document}
