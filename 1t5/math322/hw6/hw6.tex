\documentclass[letterpaper, reqno,11pt]{article}
\usepackage[margin=1.0in]{geometry}
\usepackage{color,latexsym,amsmath,amssymb,graphicx,float,listings,tikz}
\usepackage{hyperref}

\hypersetup{
colorlinks=true,
linkcolor=magenta,
filecolor=magenta,
urlcolor=cyan,
}

\graphicspath{ {images/} }

\begin{document}
\pagenumbering{arabic}
\title{Math 322 Homework 6}
\date{16/10/23}
\author{Xander Naumenko}
\maketitle

{\medskip\noindent\bf Question 2.} Consider $G,H,K$ as in the question. Then by theorem 1.5 in the textbook we have that $|G|=|H|[G:H]$, $|H|=|K|[H:K]$ and $|G|=|K|[G:K]$. Multiplying these three identities together, we get $|K|[G:K]|G||H|=|H|[G:H]|K|[H:K]\implies[G:K]=[G:H][H:K]$.

{\medskip\noindent\bf Question 3.} Let $x\in G$ and $y\in (H_1\cap H_2)x$. Then there exists $h\in H_1\cap H_2$ s.t. $hx=y$, so $h$ also witness that both $y\in H_1 x$ and $y\in H_2x$. Since this is true of any $y$ we have that $(H_1\cap H_2)x\subseteq H_1x\cap H_2 x$.

For the other direction, let $y\in H_1x\cap H_2x$. Then there exist $h_1\in H_1,h_2\in H_2$ with $y=h_1x$ and $y=h_2x$. But every element in a group is invertible so $h_1=h_2=yx^{-1}$, so in particular $h_1\in H_1\cap H_2\implies(H_1\cap H_2)x\supseteq H_1x\cap H_2 x$. Putting the two last paragraphs together we get that $(H_1\cap H_2)x= H_1x\cap H_2 x$.

To prove Poincar\'e's theorem, since we have that $[G:H_1]<\infty$ and $[G:H_2]<\infty$, we can write $G=H_1x_1\cup\ldots\cup H_1 x_m$, $G=H_2y_1\cup\ldots\cup H_2 y_n$ for some $x_i\in G, y_i\in G$ with $H_1x_i\cap H_1x_j=\emptyset,H_2y_i\cap H_2y_j=\emptyset$ for $i\neq j$. By our previously proven result, every coset $(H_1\cap H_2)z$ can be written as $H_1z\cap H_2z$, but there are only $m$ and $n$ unique cosets for $H_1$ and $H_2$ in $G$ respectively, so there are at most $mn<\infty$ unique cosets generated this way.

{\medskip\noindent\bf Question 4.} Let $G= \langle s_1,s_2,\ldots, s_n\rangle$ and assume that $H\subseteq G$ with finite index. Since $H$ has finite index we can write $G=H x_1 \cup H x_2\cup \ldots\cup H x_{n-1}$ with $x_1=1$. Thus for every combination $x_i,s_j$, we have that there exists $h_{ij},x_{k_{ij}}$ such that $x_i s_j=h_{ij}x_{k_{ij}}$. I claim that the finite set of all these $h_{ij}$s generate $H$. To see why, let $h\in H$. Since $G$ is finitely generated we can write $h=s_{i_1}\cdot\ldots s_{i_m}$. Since $x_1=1$, we can write $s_{i_1}=x_1s_{i_1}=h_{1i_{1}}x_{k_{1i_1}}$. We've thus converted our previous expression for $h$ into $h=h_{1i_1}x_{k_{1i_1}}s_{i_2}\cdots s_{i_m}$. Now considering $x_{k_{1i_1}}s_{i_2}$, we can repeat this process repeatedly to convert each element in this product to purely elements of $H$, to arrive at a product of the form $h=h_{1i_1}\cdots h_{mi_m}x_{k_{mi_m}}$. I claim that $x_{k_{mi_m}}=x_1=1$. Since $h_{1i_1}\cdots h_{mi_m}\in H$, if $x_{k_{mi_m}}\neq x_1$ then the right side of the equality wouldn't be in $H$, but since $h\in H$ it must be that the last element is $x_1$. Thus we have that $h=h_{1i_1}\cdots h_{mi_m}$ is a finite combination of the $h_{ij}$s.

{\medskip\noindent\bf Question 5.} Denote $f_{hk}(x)=hxk$ be the elements of the group described, and let $F$ be the set of all such maps. Clearly $f_{hk}$ permutes elements of $G$, so we just need to show that it is indeed a group. For closure, let $f_{hk}$ and $f_{h'k'}$ be maps and note that $f_{hk}f_{h'k'}x=hh'xk'k=f_{(hh')(k'k)}x$ which is in $F$ (since $H,K$ are subgroups $hh'\in H$ and $k'k\in K$). Note that $f_{h^{-1}k^{-1}}f_{hk}=h^{-1}hxkk^{-1}x=x$, so invertibility is fulfilled. Finally since they are subgroups $1\in H,1\in K$, so $f_{11}x=1x1=x$ for identity. Since $F$ is a group and it permutes elements of $G$, it is a group of transformations.

Consider an arbitrary combination of these maps, $f_{h_1k_1}f_{h_1k_1}\ldots f_{h_mk_m}x=h_1h_2\ldots h_mxk_m\ldots k_1$. Since $H,K$ are groups, by closure $h_1h_2\ldots h_m\in H$ and $k_m\ldots k_1\in K$, so $f_{h_1k_1}f_{h_1k_1}\ldots f_{h_mk_m}x\in HxK$. But also every element $y=hxk\in HxK$ is reachable from $x$ via $f_{hk}$, so we have that the orbit of $x$ is exactly $HxK$.

Now suppose $G$ is finite. I will prove the first equality, the second follows by the exact same argument except with right multiplication replaced with left and vice versa. Let $A=x^{-1}Hx\cap K$. I claim that there is a bijection between $K/A$ to $HxK /H$, more specifically the mapping $Ak\to Hxk$. To show that it is well defined, consider $k,k'$ such that $Ak=Ak'$. Then we have that $k(k')^{-1}\in A\implies k(k')^{-1}\in x^{-1}Hx$, which implies that $xk(k')^{-1}x^{-1}\in H\implies Hxk=Hxk'$. 

To show one-to-one, assume that for some $k,k'$ we have that $Hxk=Hxk'$. Then just applying the same logic we just used in reverse, $xk(k')^{-1}x^{-1}\in H\implies k(k')^{-1}\in x^{-1}Hx\implies k(k')^{-1}\in A$ (since also $k,k'\in K$) $\implies Ak=Ak'$. The mapping is clearly onto, since for any coset $Hxk$ of $H$ in $HxK$, $Mk$ maps to it. Thus $|A|$ is the cardinality of the number of cosets of $H$ in $HxK$ and each one has size $|H|$, so putting this together gives $|HxK|=|H||A|=|H| \left| K:x^{-1} Hx\cap K \right| $.

{\medskip\noindent\bf Question 3.} Let $g=(a,b)\in G$ and $k=(1,c)\in K$. Note that as proven in homework 2, $g^{-1}=(\frac{1}{a},-\frac{b}{a})$. Then we have that
\[
g^{-1}kg=\left( \frac{1}{a},-\frac{b}{a} \right) \left( 1,c \right) (a,b)=\left( \frac{1}{a},-\frac{b}{a} \right)\left( a,b+c \right) =(1,\frac{c}{a}-\frac{b}{a})\in K
.\]
Thus $K$ is normal. For the second part, define a map $\phi: G/K\to (\mathbb{R}^{*},\cdot,1)$ as $(a,b)K\to a$. Since multiplication by $(1,c)$ scales the second element arbitrarily, this is a well defined function as $a\in \mathbb{R}$ is the only free parameter in both sides. It is also injective and onto, since for different $a$ on the left produce different outputs and for any real $a$, choosing $(a,0)$ produces it. Thus $G /K\cong (\mathbb{R}^{*},\cdot,1)$.

{\medskip\noindent\bf Question 4.} Let $H$ be a subgroup of $G$ with index 2, for any $h\in H$, $hH=Hh$. Since $[G:H]=2$, $H'=G\setminus H$ is also a group. For any $h'\in H'$, we also have that $h'H=H'$ and $Hh'=H'$, since otherwise any element $h\in H$ with $hh'\in H$ would imply that $h'\in H'$, contradiction. Since $x\in H$ or $x\in H'$ are the only possibilities, we thus have that in general $xH=Hx\forall x\in G$. Applying $x^{-1}$ on both sides give $xHx^{-1}=H$, so $H$ is normal.

To see that $A_n$ is normal in $S_n$, all we must do is show that $[S_n:A_n]=2$, then by the previously proven property the result follows. By the previously shown result in the textbook in section 1.7 we know that $|S_n|=2|A_n|$, but we also know that by theorem 1.5 $|S_n|=[S_n:A_n]|A_n|$, which when put together give $[S_n:A_n]=2$ as required.

{\medskip\noindent\bf Question 5.} Consider normal subgroups $H_1,H_2$ and let $H=H_1\cap H_2$. Let $x\in G$. Then for all $h_1\in H_1$, $xh_1x^{-1}\in H_1$ and for all $h_2\in H_2$, $xh_2x^{-1}\in H_2$. But then for any $h\in H$ using these facts we have that $xhx^{-1}\in H_1$ and $xhx^{-1}\in H_2$, i.e. $xhx^{-1}\in H_1\cap H_2=H$, so $H$ is normal. If instead of just two normal subgroups we had a list of normal subgroups $H_1,H_2\ldots$ with $H=H_1\cap H_2\cap \ldots$, we can repeatedly apply the version just shown to reduce the problem until only a single normal subgroup remains.

Let $H,K$ be normal subgroups of $G$. Let $hk\in HK$. Then for any $x\in G$ we have $xhkx^{-1}=xhx^{-1}xkx^{-1}$. Both $xhx^{-1}\in H$ and $xkx^{-1}\in K$ by hypothesis, so we have that $xhkx^{-1}\in HK$, the requirement for $HK$ to be normal.


\end{document}
