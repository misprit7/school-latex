\documentclass[letterpaper, reqno,11pt]{article}
\usepackage[margin=1.0in]{geometry}
\usepackage{color,latexsym,amsmath,amssymb,graphicx,float,listings,tikz}
\usepackage{hyperref}

\hypersetup{
colorlinks=true,
linkcolor=magenta,
filecolor=magenta,
urlcolor=cyan,
}

\lstset{
  basicstyle=\ttfamily,
  columns=fullflexible,
  frame=single,
  breaklines=true,
  postbreak=\mbox{\textcolor{red}{$\hookrightarrow$}\space},
}

\graphicspath{ {images/} }

\begin{document}
\pagenumbering{arabic}
\title{Math 437 Homework 3}
\date{07/11/23}
\author{Xander Naumenko}
\maketitle

{\medskip\noindent\bf Question 1.} We are trying to find instances of when $a_n\equiv 0\mod 2023$, so consider the recurrence relation $\mod 2023$. All future values of the sequence are determined by the 5-tuple $(a_n,a_{n+1},a_{n+2},n^2,5^{n})$, each value within being modulo $2023$. There are $2023$ possibilities for the first 4 and $\text{ord}_{2023}5=816$ possibilities for the last, so by the pigeonhole principle the sequence must repeat after at most $2023^{4}\cdot 816\approx 1.367\cdot 10^{16}$ steps. Since it repeats infinitely from then on out, all we must do is show that there is at least one 0 within the repeating section. I claim that $a_0=0$ is in the repeating section which fulfills this requirement.

To see why, let $p$ be the period of repetition (i.e. the smallest number such that $a_m\equiv a_{m+p}, 5^{m}\equiv 5^{m+p}, m^2\equiv (m+p)^2\mod 2023$ for all $m$ sufficiently large, this is guaranteed to exist as shown above) and $n$ be the smallest number such that $a_{m}\equiv a_{m+p},5^{m}\equiv 5^{m+p},m^2=(m+p)^2\mod 2023\forall m\geq n$. In particular, $a_{n}\equiv a_{n+p},a_{n+1}\equiv a_{n+p+1},a_{n+2}\equiv a_{n+p+2}\mod 2023$. Note that $816|p$ and $2023|p$ due to the $5^{m}$ and $m^2$ requirement and the fact that $2023$ isn't a perfect square. Then consider $a_{n-1}$, using the fact that $11\cdot 184\equiv 1\mod 2023$:
\[
a_{n-1}\equiv 184\left( a_{n+2}-5^{n}a_{n+1}-n^2a_{n} \right)\equiv 184\left( a_{n+p+2}-5^{n+p}a_{n+p+1}-(n+p)^2a_{n+p} \right) \mod 2023
\]
\[
\equiv a_{n+p-1}\mod 2023
.\]
Since $5^{n-1}\equiv 1214\cdot 5^{n}\equiv 1214\cdot 5^{n+p}\equiv 5^{n+p-1}\mod 2023$ and $(n-1)^2\equiv n^2-2n+1\equiv (n+p)^2-2(n+p)+1\equiv (n+p-1)^2\mod 2023$, this contradicts our assumption that $n$ was chosen to be minimal. The only way this doesn't lead to a contradiction is if $n=0$ as $a_{n-1}=a_{-1}$ isn't defined. Thus $a_0\equiv 0\mod 2023$ is in the repetition and $2023|a_n$ infinitely many times. $\square$

While this concludes the proof, alternatively to carefully proving that $a_0$ is in the repeating section one also could have just brute force searched for a repeating 5-tuple in the form above and checked that the repeating section contains a zero. Here's some python code to do so, it turns out to repeat with period $p=4660992$ and $n=0$ as expected.

\begin{lstlisting}
N = 10000000
a = [0,1,2] + [0]*(N-3)

seen = {}
repeat_n = -1
repeat_h = ()

for n in range(0,N-3):
    a[n+3] = (5**(n%816)*a[n+2]+n**2*a[n+1]+11*a[n])%2023
    h = (n%816,(n**2)%2023,a[n+2],a[n+1],a[n])
    if h in seen:
        print(f'Found at n={n}')
        repeat_n = n
        repeat_h = h
        break
    seen[h] = n

if repeat_n == -1:
    print('No repeat found')
else:
    n1 = seen[repeat_h]
    n2 = repeat_n
    print(f'Found repeat at n1={n1}, n2={n2}')
    print(repeat_h)
    print('Searching for zeros...')
    for n in range(n1,n2+1):
        if a[n] == 0:
            print(f'Found zero at n={n}')
            print(f'Sanity check: n1={n1}, a[{n1}]={a[n1]}, a[{n1}+1]={a[n1+1]}, a[{n1}+2]={a[n1+2]}, (5^({n1}))%2023={(5**n1)%2023}, ({n1}^2)%2023={(5**n1)%2023}')
            print(f'Sanity check: n2={n2}, a[{n2}]={a[n2]}, a[{n2}+1]={a[n2+1]}, a[{n2}+2]={a[n2+2]}, (5^({n2}))%2023={(5**n2)%2023}, ({n2}^2)%2023={(5**n2)%2023}')
            break
\end{lstlisting}

{\medskip\noindent\bf Question 2.} Factor $n$ as $n=2^{\alpha_0}\prod_{i=1}^{r}p_i^{\alpha_i}$ with the $p_i$ being odd primes. Let $P(x)=x^{3}-1$, as proven in theorem 8.2, it is sufficient to find $N_P(p_i^{\alpha_i})$ and $N_P(2^{\alpha_0})$, then multiply them all together to find $N_P(n)$. Proposition 18.2 from the notes tells us that the number of solutions to $x^{m}\equiv a\mod p^{\alpha}$ is
\[
\begin{cases}
    0&\text{ if }a^\frac{\phi(p^{\alpha})}{\gcd(m,\phi(p^{\alpha}))}\not\equiv 1\mod p^{\alpha}\\
    \gcd(m,\phi(p^{\alpha}))&\text{ if }a^\frac{\phi(p^{\alpha})}{\gcd(m,\phi(p^{\alpha}))}\equiv 1\mod p^{\alpha}\\
\end{cases}
.\]
Since in this case $a=1,m=3$, this reduces to $N_P(p_i^{\alpha_i})=\gcd\left( 3,\phi(p_i^{\alpha_i}) \right) =3\text{ if }3|p_i^{\alpha_i-1}\left( p_i-1 \right), 1\text{ otherwise}$. For the $2^{\alpha_0}$ factor we can't apply Proposition 18.2 since it only applies to odd primes, so another argument is needed. Consider $\phi(2^{n})$, since half the numbers less than $2^{n}$ are even and the other half are odd, $\phi(2^{n})=2^{n-1}$. However $3\nmid 2^{n-1}\forall n$, so the only solution to $x^{3}\equiv 1\mod 2^{\alpha_0}$ is $x\equiv 1\mod 2^{\alpha_0}$. Multiplying these all together, we get that
\[
N_{P}(n)=N_P(2^{\alpha_0})\prod_{i=1}^{r}N_P(p_i^{\alpha_i})=3^{\text{\# of }i\text{ s.t. }3|p_i^{\alpha_i-1}(p_i-1)}
.\ \square\]
To double check that this correctly counts everything once again one can write a quick python script:
\begin{lstlisting}
from sympy import factorint

n = 2**3 * 3**3 * 5**2 * 7**2 * 79

def brute_force_count(n):
    count = 0
    for i in range(n):
        if i**3 % n == 1:
            count += 1
    return count

def smart_count(n):
    factors = factorint(n)
    count = 1
    for p,alpha in factors.items():
        if p == 2:
            continue
        if p**(alpha-1)*(p-1)%3 == 0:
            count *= 3
    return count

print(brute_force_count(n), smart_count(n))
\end{lstlisting}

{\medskip\noindent\bf Question 3.} I will construct the sequence $\{n_k\}_{k\geq 1}$ and show that the sequence is valid. They will be in the form $n_k=\prod_{i=1}^{k}p_i^{a_i}$, $p_i$ is the $i$th prime and $a_i\in \{0,1\}$ are shared between all $n_k$ (i.e. $a_i$ isn't a function of $k$), so the sequence $a_i$ uniquely defines the sequence $n_k$. 

Let $\alpha\in [0,1]$. Note that for $n_k$ in the form above we have $n_k=p_k^{a_k}n_{k-1}$. Let $a_k=1$ if $\left( 1-\frac{1}{p_k} \right) \prod_{i=1}^{k-1}\left(1-\frac{1}{p_i}\right)^{a_i}\geq \alpha$ and $a_k=0$ otherwise. I claim that these $a_k$ define a sequence $n_k$ that fulfills the required limit.

If $\alpha=1$ then $n_k=1$ works, so assume $\alpha<1$. Let $\epsilon>0$, and to make some of the algebra later simpler assume that $\epsilon<1-\alpha$. Expanding out the given limit using the fact that $a_i\in \{0,1\}$, we have
\[
\lim_{k\to\infty}\frac{\phi(n_k)}{n_k}=\prod_{i=1}^{\infty}\left( \frac{p_i-1}{p_i} \right) ^{a_i}=\prod_{i=1}^{\infty}\left( 1-\frac{1}{p_i} \right) ^{a_i}
.\]
The product is decreasing and the $a_i$s were specifically constructed so that each of these partial products are always greater than $\alpha$, so all that needs to be shown is that there is some partial product that is smaller than $\alpha+\epsilon$. Before this can be done, first note that $\prod_{i=1}^{\infty}\left( 1-\frac{1}{p_i} \right) =0$ (this relies on some slightly non-rigorous expansion but something very similar was shown in class so I assume it's fine):
\[
    \prod_{i=1}^{\infty}\left( 1-\frac{1}{p_i} \right) =\left(\prod_{i=1}^{\infty}\frac{1}{1-p_i^{-1}}\right)^{-1}=\left(\prod_{i=1}^{\infty}1+p_i^{-1}+p_i^{-2}+\ldots\right)^{-1}=\left( \sum_{i=1}^{\infty}\frac{1}{i} \right) ^{-1}=\frac{1}{\infty}=0
.\]

Let $K$ be such that $\frac{1}{p_K}<\epsilon$ and denote $A=\prod_{i=1}^{K}\left( 1-\frac{1}{p_i} \right) ^{a_i}$. If $A<\alpha+\epsilon$ already then we're done, so assume it isn't. Then since $\prod_{i=1}^{\infty}\left( 1-\frac{1}{p_i} \right) =0$ there exists an $N$ such that $B=A\prod_{i=K+1}^{N}\left( 1-\frac{1}{p_i} \right)<\alpha+\epsilon$, also assume that $N$ is chosen to be the smallest possible that fulfills this property. By the minimality of $N$ we have that $B\left( 1-\frac{1}{p_N} \right) ^{-1}\geq \alpha+\epsilon$, so using the fact that $\alpha+\epsilon\leq 1$ we get
\[
B\geq \left( \alpha+\epsilon \right) \left( 1-\frac{1}{p_N} \right)\geq \left( \alpha+\epsilon \right) \left( 1-\frac{1}{p_K} \right)\geq \left( \alpha+\epsilon \right) (1-\epsilon)\geq \alpha+\epsilon-\epsilon=\alpha
.\]

Thus $B\in [\alpha, \alpha+\epsilon)$. Then by the construction of $a_i$ we have that $a_{K+1}=\ldots=a_{N}=1$, so $B=A\prod_{i=K+1}^{N}\left( 1-\frac{1}{p_i} \right)=\prod_{i=1}^{N}\left( 1-\frac{1}{p_i} \right) ^{a_i}=\frac{\phi(n_N)}{n_N}\in [\alpha,\alpha+\epsilon)$. Since $\frac{\phi(n_k)}{n_k}$ is bounded below by $\alpha$, is decreasing and comes arbitrarily close to $\alpha$, its limit is $\alpha$. $\square$



\end{document}
