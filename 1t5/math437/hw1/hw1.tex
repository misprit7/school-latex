\documentclass[letterpaper, reqno,11pt]{article}
\usepackage[margin=1.0in]{geometry}
\usepackage{color,latexsym,amsmath,amssymb,graphicx,float,listings,tikz}
\usepackage{hyperref}

\hypersetup{
colorlinks=true,
linkcolor=magenta,
filecolor=magenta,
urlcolor=cyan,
}

\graphicspath{ {images/} }

\begin{document}
\pagenumbering{arabic}
\title{Math 437 Homework 1}
\date{28/09/23}
\author{Xander Naumenko}
\maketitle

{\medskip\noindent\bf Question 1.} Let $m\in \mathbb{N}$, and let $s_i=\frac{i}{\sqrt{i+1}}$. $s_i$ is clearly unbounded since $\frac{i}{\sqrt{i+1}}= \frac{i+1}{\sqrt{i+1}}-\frac{1}{\sqrt{i+1}}\geq\sqrt{i+1}-1$ is unbounded. Also note that $\frac{s_{i+1}}{s_i}=\frac{(i+1)\sqrt{i+1}}{i\sqrt{i+2}}\geq \frac{i+1}{i}>1$, so $s_i$ is strictly increasing. Thus there exists a smallest number $n\in\mathbb{N}$ with $m\leq s_n$.

I claim that $m=\left[\frac{n}{\sqrt{n+1}}\right]$. To see why suppose by contradiction not, since $m\leq \frac{n}{\sqrt{n+1}}$ this would mean $\frac{n}{\sqrt{n+1}}\geq m+1$. But then $s_{n-1}=\frac{n-1}{\sqrt{n}}\geq \frac{n-\sqrt{n+1}}{\sqrt{n+1}}\geq m$, which contradicts the assumption that $n$ was the smallest possible. Thus it must be that $m=\left[\frac{n}{\sqrt{n+1}}\right]$.

{\medskip\noindent\bf Question 2.} Let $A$ be the union of finitely many arithmetic progressions with coefficients $(a_1,b_1), \ldots (a_N,b_N)$ respectively. Let $m=\text{lcm}\left( a_1,a_2,\ldots,a_N \right) $ and $B=\max(b_1, b_2, \ldots b_N) $. Then for each $k\in \{B,B+1,\ldots B+m-1\}$, if $k\in S$ then define an arithmetic progression with $a_k'=m,b_k'=k$. Let $S'$ be the union of all arithmetic progressions generated this way, along with the finite set $\{n\in S:n<B\}$.

I claim $S=S'$. Let $c\in S$, if $c< B$ then by construction $c\in S'$. Otherwise, by the division algorithm let $c-B=mn+r, n\geq0, 0\leq r<m$. $B+r\in S$, since if it wasn't then the progression $(a_i,b_i)$ with $B+r = a_i n'+b_i$ would also include $c$ since $c=(B+r)+mn$ and $a_i|m$. Thus $S\subseteq S'$.

For the other direction, let $d\in S'$. Again if $d< B$ then by construction $d\in S$ automatically. Otherwise, let $a_i,b_i$ be two coefficients used to generate $S$ originally, and since $d\in S'$ it can be written as $d=mn+B+r$ for some $r\in \{0,1,\ldots,m-1\}$. Since there exists no $n\geq 0$ with $a_i n+b_i=B+r$ by the definition of $S'$ and $a_i|m$, there also clearly can't be a solution to $d=a_i n+b_i$. Thus $d\in S$, so $S'\subseteq S$. Since both sets contain one another, $S=S'$ as required.

{\medskip\noindent\bf Question 3.} 

\end{document}
