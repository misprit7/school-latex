\documentclass[letterpaper, reqno,11pt]{article}
\usepackage[margin=1.0in]{geometry}
\usepackage{color,latexsym,amsmath,amssymb,graphicx,float,listings,tikz}
\usepackage{hyperref}

\hypersetup{
colorlinks=true,
linkcolor=magenta,
filecolor=magenta,
urlcolor=cyan,
}

\graphicspath{ {images/} }

\begin{document}
\pagenumbering{arabic}
\title{Math 406 Homework 2}
\date{10/10/23}
\author{Xander Naumenko}
\maketitle

{\medskip\noindent\bf Question 1.} See tables \ref{tab:q1a}, \ref{tab:q1b}, \ref{tab:q1c}, \ref{tab:q1d} and \ref{tab:q1e} for the required tables. For the log-log plots see figures \ref{fig:q1a}, \ref{fig:q1b}, \ref{fig:q1c}, \ref{fig:q1d} and \ref{fig:q1e}.

The code for this question can be seen here:
\begin{lstlisting}
disp('midpoint')
numerical_integration(-1, 1, @(x) 1/(1+x^2)^0.5, 32, 1);
disp('trap')
numerical_integration(-1, 1, @(x) 1/(1+x^2)^0.5, 32, 2);
disp('simpson')
numerical_integration(-1, 1, @(x) 1/(1+x^2)^0.5, 32, 3);
disp('gauss_legendre')
numerical_integration(-1, 1, @(x) 1/(1+x^2)^0.5, 32, 4);
disp('real')
% -2*log(2^0.5-1)

functions = {@(x) 1/(1+x^2)^0.5, @(x) sin(2*x)^2, @(x) x^(4.0/3), @(x) x^(1.0/3), @(x) (-log(x))^0.5};
bounds = [-1,1; 0,pi; 0,1; 0,2; 0,1];
real_ans = [-2*log(2^0.5-1), pi/2, 3.0/7, 3/(2^(2.0/3)), pi^0.5/2];
Ns = [2,4,16,32];
tables = zeros(5,4,5);
for i = 1:length(functions)
    f = functions{i};
    for j = 1:length(Ns)
        N = Ns(j);
        for choice = [1,2,3,4]
            if i ~= 5 || (choice ~= 2 && choice ~= 3)
                tables(i,j,choice) = numerical_integration(bounds(i,1),bounds(i,2),f,N,choice);
            end
        end
        tables(i,j,5) = real_ans(i);
    end
end

for i = 1:5
    disp(array2table(squeeze(tables(i,:,:))))
end


% Calculate the errors
errors = abs(tables - real_ans');

methods = {'Midpoint', 'Trapezium', 'Simpson', 'Gauss-Legendre'};
colors = {'r', 'g', 'b', 'k'};

for func_idx = 1:5
    figure('Name', ['Function ' num2str(func_idx)]); % Creates a new figure for each function
    hold on;
    for choice = 1:4
        loglog(Ns, squeeze(errors(func_idx, :, choice)), '-o', 'Color', colors{choice}, 'DisplayName', methods{choice});
    end
    xlabel('N');
    ylabel('Error');
    title(['Log-Log plot of N vs Error for Function ', num2str(func_idx)]);
    
    % Ensuring that the axes are in log-log scale
    set(gca, 'XScale', 'log', 'YScale', 'log');
    
    legend('Location', 'southwest');
    grid on;
    hold off;
end

% 1. Midpoint rule with N cells
% 2. Trapezium rule with N cells
% 3. Simpson''s rule with 2N cells
% 4. Three-point Gauss-Legendre quadrature with N cells
function result = numerical_integration(a, b, f, N, choice)

    switch choice
        case 1
            result = midpoint_rule(f, a, b, N);
        case 2
            result = trapezium_rule(f, a, b, N);
        case 3
            result = simpsons_rule(f, a, b, N);
        case 4
            result = gauss_legendre(f, a, b, N);
        otherwise
            return;
    end

    % fprintf('The result of the integration is: %.5f\n', result);
end

function result = midpoint_rule(f, a, b, N)
    result = 0;
    for i = 0:(N-1)
        point = a+(i+0.5)*(b-a)/N;
        result = result + f(point);
    end

    result = result * (b-a)/N;
end


function result = trapezium_rule(f, a, b, N)
    result = 0;
    for i = 1:(N-1)
        point = a+i*(b-a)/N;
        result = result + 2*f(point);
    end
    result = result + f(a) + f(b);

    result = result * (b-a)/N/2;
end

function result = simpsons_rule(f, a, b, N)
    result = 0;
    for i = 1:N
        left = a+(i-1)*(b-a)/N;
        right = a+i*(b-a)/N;
        result = result + 1/3*(b-a)/N/2*(f(left)+4*f((left+right)/2)+f(right));
    end
end

function result = gauss_legendre(f, a, b, N)
    x = [-sqrt(3/5), 0, sqrt(3/5)];
    w = [5/9, 8/9, 5/9];

    result = 0;
    for i = 1:N
        left = a + (i-1)*(b-a) / N;
        right = a + i*(b-a)/N;
        for j = 1:3
            xi = 0.5*(left+right + (right-left) * x(j));
            result = result+w(j)*f(xi);
        end
    end
    result = 0.5 * (b-a)/N*result;
end
\end{lstlisting}

\begin{table}
\centering
\begin{tabular}{|c|c|c|c|c|}
\hline
$N$ & Midpoint & Trapezium & Simpson & Gauss-Legendre \\
\hline
2 & 1.78885438199983 & 1.70710678118655 & 1.76160518172874 & 1.76266240387583 \\
\hline
4 & 1.77014250014533 & 1.74798058159319 & 1.76275519396128 & 1.76274697462844 \\
\hline
16 & 1.76320768598367 & 1.7618262836833 & 1.76274721855022 & 1.76274717401768 \\
\hline
32 & 1.76286227284615 & 1.76251698483349 & 1.76274717684193 & 1.76274717403875 \\
\hline
\end{tabular}
\caption{Question 1a}
\label{tab:q1a}
\end{table}

\begin{table}
\centering
\begin{tabular}{|c|c|c|c|c|}
\hline
$N$ & Midpoint & Trapezium & Simpson & Gauss-Legendre \\
\hline
2 & 3.14159265358979 & 7.06745147303987e-32 & 2.0943951023932 & 1.60606730241802 \\
\hline
4 & 1.5707963267949 & 1.5707963267949 & 1.5707963267949 & 1.5707963267949 \\
\hline
16 & 1.5707963267949 & 1.5707963267949 & 1.5707963267949 & 1.5707963267949 \\
\hline
32 & 1.5707963267949 & 1.5707963267949 & 1.5707963267949 & 1.5707963267949 \\
\hline
\end{tabular}
\caption{Question 1b}
\label{tab:q1b}
\end{table}

\begin{table}
\centering
\begin{tabular}{|c|c|c|c|c|}
\hline
$N$ & Midpoint & Trapezium & Simpson & Gauss-Legendre \\
\hline
2 & 0.419455176774456 & 0.448425131496025 & 0.429111828348312 & 0.428525592513674 \\
\hline
4 & 0.426049167558007 & 0.43394015413524 & 0.428679496417085 & 0.428562331914328 \\
\hline
16 & 0.428391866554757 & 0.428943359698886 & 0.4285756976028 & 0.428571070406974 \\
\hline
32 & 0.428524607238247 & 0.428667613126822 & 0.428572275867772 & 0.428571357502592 \\
\hline
\end{tabular}
\caption{Question 1c}
\label{tab:q1c}
\end{table}

\begin{table}
\centering
\begin{tabular}{|c|c|c|c|c|}
\hline
$N$ & Midpoint & Trapezium & Simpson & Gauss-Legendre \\
\hline
2 & 1.93841476853743 & 1.62996052494744 & 1.83559668734077 & 1.89373800719319 \\
\hline
4 & 1.91040464923353 & 1.78418764674243 & 1.86833231506983 & 1.89141202322811 \\
\hline
16 & 1.89332086475362 & 1.8728210349583 & 1.88648758815518 & 1.89012260552418 \\
\hline
32 & 1.89126652809474 & 1.88307094985596 & 1.88853466868182 & 1.8899772279319 \\
\hline
\end{tabular}
\caption{Question 1d}
\label{tab:q1d}
\end{table}

\begin{table}
\centering
\begin{tabular}{|c|c|c|c|c|}
\hline
$N$ & Midpoint & Trapezium & Simpson & Gauss-Legendre \\
\hline
2 & 0.856885021909063 & 0 & 0 & 0.881394330687781 \\
\hline
4 & 0.870845677383917 & 0 & 0 & 0.883847456341008 \\
\hline
16 & 0.882289474604227 & 0 & 0 & 0.885658742625317 \\
\hline
32 & 0.884274758802113 & 0 & 0 & 0.88595040876536 \\
\hline
\end{tabular}
\caption{Question 1e}
\label{tab:q1e}
\end{table}

\begin{figure}[htpb]
    \centering
    \includegraphics[width=0.8\textwidth]{q1a}
    \caption{Error of Question 1a.}
    \label{fig:q1a}
\end{figure}
\begin{figure}[htpb]
    \centering
    \includegraphics[width=0.8\textwidth]{q1b}
    \caption{Error of Question 1b. Note that due to the fact that the error goes to zero very quickly for the different methods by chance means this plot doesn't look like the others.}
    \label{fig:q1b}
\end{figure}
\begin{figure}[htpb]
    \centering
    \includegraphics[width=0.8\textwidth]{q1c}
    \caption{Error of Question 1c.}
    \label{fig:q1c}
\end{figure}
\begin{figure}[htpb]
    \centering
    \includegraphics[width=0.8\textwidth]{q1d}
    \caption{Error of Question 1d.}
    \label{fig:q1d}
\end{figure}
\begin{figure}[htpb]
    \centering
    \includegraphics[width=0.8\textwidth]{q1e}
    \caption{Error of Question 1e. Ignore the error for trapezium/Simpson, they were set to 0 in the code so had a constant error.}
    \label{fig:q1e}
\end{figure}

{\medskip\noindent\bf Question 2.} Using the given asymptotic expansion for the Trapezium rule error, we get
\[
I(0)-I(h_{s+1})=I(0)-I(\frac{1}{2}h_s)=\sum_{i=1}^{\infty}\frac{1}{2^{2i}}c_ih_s^{2i}
\]
\[
\implies 4I(\frac{1}{2}h_s)=4I(0)-\sum_{i=1}^{\infty}\frac{1}{2^{2i}}c_ih_s^{2i}
\]
\[
\implies I(0)=\frac{4I(\frac{h_2}{2})-I(h_2)}{3}+ \sum_{i=2}^{\infty}\frac{2^{2(i-1)}-1}{3\cdot 2^{2(i-1)}}c_i h_s^{2i}
.\]
Define $a_s^{(1)}=I(h_s)$ and $c_i^{(2)}$ as in the question. Then simply rearranging the above formula algebraically, we get
\[
I(0)-\left(I(\frac{h_2}{2})+\frac{I(\frac{h_2}{2})-I(h_s)}{3}\right)=I(0)-a_{s+1}^{(1)}-\frac{a_{s+1}^{(1)}-a_s^{(1)}}{3}=I(0)-a_{s}^{2}=\sum_{i=2}^{\infty}c_{i}^{(2)}h_s^{2i}
\]
as required. To eliminate the $O(h^{4})$ term we can again rearrange, this time leaving off some algebra and leaving $c_i^{(3)}$ to be defined in the next part:
\[
I(0)-a_{s+1}^{2}+ \frac{a_{s+1}^{2}-a_s^{m-1}}{15}=\sum_{i=3}^{\infty}c_i^{(3)}h_s^{2i}
.\]

To find the general recursion formula, we follow a similar process to what we just did. We've already shown the result $m=2$, using recursion we just need to prove that the result holds for $m$ given that it holds for $m-1$. Assume that for some $m$,
\[
    I(0)-a_{s}^{(m-1)}=\sum_{i=m-1}^{\infty}c_i^{(m-1)}h_s^{2i}
\]
and
\[
    I(0)-a_{s+1}^{(m-1)}=\sum_{i=m-1}^{\infty}\frac{1}{2^{2i}}c_i^{(m-1)}h_{s}^{2i}
.\]
Subtracting these, we get
\[
I(0)-\frac{(4^{m-1})a_{s+1}^{(m-1)}-a_s^{(m-1)}}{4^{m-1}-1}=\sum_{i=m}^{\infty}\left((\frac{1}{4^{i-1}}-1)\frac{1}{4^{m-1}-1}c_i^{(m-1)}\right)h_s^{2i}
.\]
Define $c_i^{(m)}=\left((\frac{1}{4^{i-1}}-1)\frac{1}{4^{m-1}-1}c_i^{(m-1)}\right)$. Then the previous equation is equivalent to
\[
I(0)-a_{s+1}^{(m-1)}-\frac{a_{s+1}^{(m-1)}-a_s^{(m-1)}}{4^{m-1}-1}=\sum_{i=m}^{\infty}c_{i}^{m}h_s^{2i}
.\]
Thus clearly we have that $a_s^{(m)}=a_{s+1}^{(m-1)}+\frac{a_{s+1}^{(m-1)}-a_s^{(m-1)}}{4^{m-1}-1}$ as required.


{\medskip\noindent\bf Question 3.} Subtracting the first term of the Taylor series is already done for us in the question, so we just need to subtract the three term. $\cos x=1-\frac{x^2}{2}+\ldots$, so we want to evaluate
\[
I=\int_0^{\pi/2}x^{-\frac{1}{2}}dx-\int_0^{\pi/2}\frac{1}{2}x^{\frac{3}{2}}dx+\int_0^{\pi/2}x^{-\frac{1}{2}}\left( \cos x-1+\frac{x^2}{2} \right) dx
\]
\[
=(2\pi)^{\frac{1}{2}}-\frac{\pi^{5 /2}}{20\sqrt{2}}+\int_0^{\pi/2}x^{-\frac{1}{2}}\left( \cos x-1+\frac{x^2}{2} \right) dx
.\]

Plugging in each of the methods previously coded for question 1, we get table \ref{tab:q3}. This is the code used to generate the table:
\begin{lstlisting}
f1 = @(x)x^(-0.5)*cos(x);
f2 = @(x)x^(-0.5)*(cos(x)-1);
f3 = @(x)x^(-0.5)*(cos(x)-1+x^2/2);

c1 = (2*pi)^0.5
c2 = (2*pi)^0.5-pi^(5/2)/20/2^0.5

N1 = 2^(4);
N2 = 2^(6);
a = eps^0.5;
b = pi/2;

T = zeros(8,2);

T(1,1) = midpoint_rule(f1,a,b,N1)
T(1,2) = midpoint_rule(f1,a,b,N2)

T(2,1) = gauss_legendre(f1,a,b,N1)
T(2,2) = gauss_legendre(f1,a,b,N2)

T(3,1) = midpoint_rule(f2,a,b,N1)+c1
T(3,2) = midpoint_rule(f2,a,b,N2)+c1

T(4,1) = gauss_legendre(f2,a,b,N1)+c1
T(4,2) = gauss_legendre(f2,a,b,N2)+c1

T(5,1) = trapezium_rule(f2,a,b,N1)+c1
T(5,2) = trapezium_rule(f2,a,b,N2)+c1

T(6,1) = midpoint_rule(f3,a,b,N1)+c2
T(6,2) = midpoint_rule(f3,a,b,N2)+c2

T(7,1) = gauss_legendre(f3,a,b,N1)+c2
T(7,2) = gauss_legendre(f3,a,b,N2)+c2

T(8,1) = trapezium_rule(f3,a,b,N1)+c2
T(8,2) = trapezium_rule(f3,a,b,N2)+c2


function result = midpoint_rule(f, a, b, N)
    result = 0;
    for i = 0:(N-1)
        point = a+(i+0.5)*(b-a)/N;
        result = result + f(point);
    end

    result = result * (b-a)/N;
end


function result = trapezium_rule(f, a, b, N)
    result = 0;
    for i = 1:(N-1)
        point = a+i*(b-a)/N;
        result = result + 2*f(point);
    end
    result = result + f(a) + f(b);

    result = result * (b-a)/N/2;
end

function result = simpsons_rule(f, a, b, N)
    result = 0;
    for i = 1:N
        left = a+(i-1)*(b-a)/N;
        right = a+i*(b-a)/N;
        result = result + 1/3*(b-a)/N/2*(f(left)+4*f((left+right)/2)+f(right));
    end
end

function result = gauss_legendre(f, a, b, N)
    x = [-sqrt(3/5), 0, sqrt(3/5)];
    w = [5/9, 8/9, 5/9];

    result = 0;
    for i = 1:N
        left = a + (i-1)*(b-a) / N;
        right = a + i*(b-a)/N;
        for j = 1:3
            xi = 0.5*(left+right + (right-left) * x(j));
            result = result+w(j)*f(xi);
        end
    end
    result = 0.5 * (b-a)/N*result;
end
\end{lstlisting}

\begin{table}
\centering
\begin{tabular}{|l|c|c|}
\hline
Integration Rule & \( h = \left(\frac{1}{2}\right)^2 \) & \( h = \left(\frac{1}{2}\right)^6 \) \\
\hline
Direct Midpoint & 1.765666283681793 & 1.860155868481409 \\
\hline
Direct 3 pt Gauss & 1.876838968734573 & 1.915870385659986 \\
\hline
Subtract 1 term Midpoint & 1.955096450266878 & 1.954915723357975 \\
\hline
Subtract 1 term 3 pt Gauss & 1.954903132540786 & 1.954902857457019 \\
\hline
Subtract 1 term Trapezium & 1.954504413847847 & 1.954876746976996 \\
\hline
Subtract 3 terms Midpoint & 1.954743822379750 & 1.954892907353329 \\
\hline
Subtract 3 terms 3 pt Gauss & 1.954902848557130 & 1.954902848582609 \\
\hline
Subtract 3 terms Trapezium & 1.955220931501738 & 1.954922731169139 \\
\hline
\end{tabular}
\caption{Question 3.}
\label{tab:q3}
\end{table}

{\medskip\noindent\bf Question 4.} 

\end{document}
