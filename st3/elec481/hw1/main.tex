\documentclass[letterpaper, reqno,11pt]{article}
\usepackage[margin=1.0in]{geometry}
\usepackage{color,latexsym,amsmath,amssymb,graphicx, float}
\usepackage{hyperref}

\hypersetup{
colorlinks=true,
linkcolor=magenta,
filecolor=magenta,
urlcolor=cyan,
}

\graphicspath{ {images/} }

\begin{document}
\pagenumbering{arabic}
\title{ELEC 481 Homework 1}
\date{19/05/22}
\author{Xander Naumenko}
\maketitle

{\noindent\bf Question 1a.} The net cost of installing the stainless steel pump is $\$1800$, since that's how much it costs to install it. The net cost of the brass pump is $\$6000-\$3500=\$2500$, since we recoup $\$3500$ of the cost by reselling the old pump. 

{\noindent\bf Question 1b.} The stainless steel pump is better, since it has a lower net cost. 

{\noindent\bf Question 1c.} The opportunity cost would be $\$-2500$, as that's the net benefit of the next best option. 

{\noindent\bf Question 1d.} Subtracting the values in part a, it is $\$700$ cheaper to use the stainless steel pump rather than the brass pump. 

{\noindent\bf Question 2.} The power scale model means that: 
\[
    \left( \frac{\text{cost}}{11000000} \right) = \left( \frac{6.4}{5} \right)^{0.72}\implies\text{cost}=13139652
.\]
Then in current dollars it is $\$13139652\cdot \frac{180}{110}=21.5$ million. 

{\noindent\bf Question 3.} The time value of money describes the fact that money is worth more at the present than the future. This means that you cannot just take any two monetary values and compare them even if you ignore inflation, since if they have different time scales then they are not equivalent. One example from my own life is money to buy a new phone when I was in high school. Back then I simply didn't have the funds available whereas now I do and would happily spend it on a better phone, but since I didn't have access to the money back then it would have been worth considerably more. Another example is if I ever start a startup in the future, as when starting a business money in the present is far more valuable then money down the line. 

{\noindent\bf Question 4a.} 
\[
F=\frac{P}{(1+i)^{n}}=\frac{250000}{1.035^{5}}=\$210493
.\]
{\noindent\bf Question 4b.} 
\[
F=\frac{P}{(1+i)^{n}}=\frac{250000}{1.035^{10}}=\$177230
.\]
{\noindent\bf Question 4c.} 
\[
F=\frac{P}{(1+i)^{n}}=\frac{250000}{1.035^{20}}=\$125641
.\]
{\noindent\bf Question 4d.} 
\[
F=\frac{P}{(1+i)^{n}}=\frac{250000}{1.035^{50}}=\$44763
.\]

{\noindent\bf Question 5a.} The nominal interest rate is $(\frac{\$110}{\$85}-1)\cdot 2= 59\%$. 

{\noindent\bf Question 5b.} The effective annual interest rate is $(1+0.588 /2)^2-1= 67\%$. 

\end{document}
